\mode*

\section{Scope and aims}%
\label{sec:aim}

\mode<presentation>{%
  \subsection{Scope}

  \begin{frame}
    \begin{itemize}

      \item We treat computer security.
      \item \Ie how to secure computer systems.

        \pause

      \item We cover authentication and access control, software security, 
        \etc.

        \pause

      \item This gives a good foundation that can later be expanded with things 
        like network security and information security.

    \end{itemize}
  \end{frame}
}

This course is an introduction to computer security.
The course aims towards a good understanding for the requirements of a secure 
computer system.
Problems such as authentication and access control; software security, such as 
buffer overflows; as well as operating system, library and application security 
mechanisms are treated in the course.

\mode<presentation>{%
  \subsection{Aims}
}

More concretely, the \acp{ILO} of the course are the following.
After completing the course, you should be able to:
\begin{frame}
\mode<presentation>{You should be able to}
\begin{itemize}

  \item \emph{apply} different cryptosystems and \emph{explain} how these work,

  \item \emph{analyse} problems of authentication, access control and different 
    solutions,

  \item \emph{explain} how some common attacks on software works,

  \item \emph{analyse} different operating system security mechanisms,

  \item \emph{analyse} the functionality of different types of malware,

  \item \emph{explain} different malware protection mechanisms,

  \item \emph{evaluate} strengths and weaknesses of hardware-based security and 
    full-disk encryption, as well as

  \item \emph{value and argue} about different ethical aspects of computer 
    security.

\end{itemize}
\end{frame}
The course has a variety of learning sessions designed to ensure that you learn 
these \acp{ILO}.
Each such session has a set of further specified \acp{ILO}:
\eg the first outcome above refers to \enquote{different cryptosystems}, the 
\acp{ILO} of a learning session would specify which cryptosystems it covers.


\section{Course structure and content overview}%
\label{sec:outline}

% - Basic cryptography: block ciphers, asymmetric cryptosystems, digital 
%   signatures, hash functions.
% - User authentication, access control and intrusion detection.
% - Software security: buffer overflow, handling input, interaction between 
%   programs.
% - Operating system security.
% - Malware.
% - Trusted computing, hardware based security and whole disk encryption.
% - Ethical aspects of computer security.

The course covers
applied cryptography used in computer security, \eg uses of cryptography for 
code obfuscation or digital rights management;
authentication mechanisms, access control, and intrusion detection;
software security, \eg buffer overruns and interaction between programs;
some security mechanisms provided by operating system and hardware;
and malicious software and how these utilise weaknesses in the system.
Finally, we discuss some ethical implications for computer engineers.

\subsection{Teaching and tutoring}

\mode<presentation>{%
  \begin{frame}
    \begin{itemize}

      \item Teaching consists of several types of learning sessions.
      \item Most topics are covered only by lectures.
        
        \pause

      \item Some are complemented with seminars, hand-ins and labs.
      \item These are for combined learning and assessment.
      \item We have hackathon labs.

    \end{itemize}
  \end{frame}

  \begin{frame}
    \begin{itemize}

      \item Assignments are labelled in consecutive order.
      \item Labs are prefixed with \enquote{L}.
      \item Seminars are prefixed with \enquote{S}.
      \item Memos are prefixed with \enquote{M}.
      \item Hackathons are prefixed with \enquote{H}.

    \end{itemize}
  \end{frame}
}

The course consists of several types of learning sessions:
lectures, seminars, laboratory work.
All assignments are numbered consecutively prefixed with an \enquote{L} for 
laboratory assignments, \enquote{S} for seminar assignments, \enquote{M} for 
memos and \enquote{H} for hackathons.

\subsection{Schedule}

You will find an outline for a schedule for the course in \cref{Schedule}.
You are free to follow this schedule or any schedule you make for yourself, but 
the learning and tutoring sessions, deadlines \etc will follow this schedule.
The detailed reading instructions for each item in the schedule can be found in 
the following sections.

\begin{frame}[allowframebreaks]
\mode<article>{%
\begin{table}
}
	\centering
  \begin{tabular}{lp{9cm}}
    \toprule
    \textbf{Week}	& \textbf{Work} \\
    \midrule
    1
    & Lecture: Course start/Introduction\\
    & Seminar: S0 What's up with security?\\
    \midrule
    2
    & Lecture: Foundations of security\\
    & Lecture: Security usability\\
    \midrule
    3
    & Lecture: Information theory\\
    & Lecture: Cryptography, part I\\
    & Lecture: Cryptography, part II\\
    \midrule
\mode<presentation>{%
  \end{tabular}
  \begin{tabular}{lp{9cm}}
}%
    4
    & Lecture: Identification and authentication, part I\\
    & Lecture: Identification and authentication, part II\\
    & Lab: L1 (pwdguess), L2 (pricomlab)\\
    \midrule
    5
    & Lecture: Protocols and formal verification\\
    & Seminar: S3 (pwdpolicies)\\
    \midrule
    6
    & Lecture: Access control\\
    & Lecture: Reference monitors\\
    & Lecture: Accountability\\
    & Lab: L1 (pwdguess), L2 (pricomlab), L7 (tools)\\
    \midrule
    7
    & Lecture: Trusted computing\\
    & Lecture: Software security\\
    \midrule
\mode<presentation>{%
  \end{tabular}
  \begin{tabular}{lp{9cm}}
}%
    8
    & Hackathon: H4 (drmlab)\\
    & Hackathon: H5 (stacksmash)\\
    & Lab: L1 (pwdguess), L2 (pricomlab), L7 (tools)\\
    \midrule
    9
    & Hackathon: H6 (malwarelab)\\
    & Seminar: L7 (tools)\\
    & Lecture: Course conclusion\\
    \midrule
    10
    & Exam: course exam\\
    & Lab: final call for labs\\
    \midrule
\mode<presentation>{%
  \end{tabular}
  \vspace*{1cm}
  \begin{tabular}{lp{9cm}}
}%
    +3 months
    & Exam: re-exam\\
    & Seminar: second call for seminars\\
    \midrule
    +6 months
    & Exam: last re-exam until next year\\
    & Seminar: final call for seminars\\
    \bottomrule
  \end{tabular}
\mode<article>{%
  \caption{%
    A summary of the parts of the course and when they will (or should) be done.
    The table is adapted to taking this course at half-time pace, \ie 20 hours 
    per week for 10 weeks.
  }\label{Schedule}
\end{table}
}


\end{frame}


\section<article>{Course content}

This section summarizes the material covered by the lectures and assignments, 
\ie what you should read for each of them.
It is divided by topics and ordered according to progression of the course, 
\cref{Schedule} gives an overview along with a schedule.

\mode<article>{%
\subsection{S0 What's up with security?}
\input{security-society-abstract.tex}
}

\mode<article>{%
\subsection{Foundations of security}
\input{foundations-lit.tex}
}

\mode<article>{%
\subsection{Security usability}
\input{usability-lit.tex}
}

\mode<article>{%
\subsection{Information theory}
\input{infotheory-lit.tex}
}

\mode<article>{%
\subsection{Cryptography}
\input{crypto-lit.tex}
}

\mode<article>{%
\subsection{Identification and authentication}
\input{auth-lit.tex}
}

\mode<article>{%
\subsection{L1 Password cracking and social engineering}
\input{pwdguess-lit.tex}
}

\mode<article>{%
\subsection{L2 Private communication}
\input{pricomlab-lit.tex}
}

\mode<article>{%
\subsection{S3 Password policies}
\input{pwdpolicies-lit.tex}
}

\mode<article>{%
\subsection{Protocols and formal verification}
\input{fverif-lit.tex}
}

\mode<article>{%
\subsection{Access control}
\input{accessctrl-lit.tex}
}

%\subsection{Multi-Level and Multi-Lateral Security}
%\input{lvlltrl-lit.tex}

\mode<article>{%
\subsection{Reference monitors}
\input{refmon-abstract.tex}
}

\mode<article>{%
\subsection{Accountability}
\input{accountability-lit.tex}
}

\mode<article>{%
\subsection{Trusted computing}
\input{trustcomp-lit.tex}
}

\mode<article>{%
\subsection{Software security}
\input{software-lit.tex}
}

\mode<article>{%
\subsection{L4 Tools of the trade}
\input{tools-abstract.tex}
}

\mode<article>{%
\subsection{H5 Digital rights management}
\input{drm-abstract.tex}
}

\mode<article>{%
\subsection{H6 Smashing the stack}
\input{stacksmash-abstract.tex}
}

\mode<article>{%
\subsection{H7 Malicous software}
\input{malware-abstract.tex}
}

\mode<article>{%
\subsection{Course conclusion}
}

During this lecture we will shortly review the course and try to fit things into 
a bigger picture.

\mode<article>{%
\subsection{Final exam}

The final exam will assess how well you have achieved the intended learning 
outcomes of the course.
Hence, it covers all the content given above.

Each question on the exam covers one topic of the course.
To pass the exam (and thus the course) you must pass all questions (thus all 
topics), \ie you must not receive zero on any question.

If you receive a zero on one question you qualify for an oral
}


\section{Assessment}%
\label{Assessment}

\mode<presentation>{%
  \subsection{LADOK modules}
}

This section explains how the course modules are graded and mapped to LADOK\@.
\Cref{LADOKTable} visualizes the relations between modules, credits, grades and 
LADOK\@.

\begin{frame}
  \begin{table}
  \centering
  \setlength{\tabcolsep}{0.5em}
  \begin{tabular}{rrll}
    \toprule
    \textbf{LADOK}
    & \textbf{Credits (ECTS)}
    & \textbf{Grade}
    & \textbf{Course Assignments}
    \\
    \midrule
    I104  & 0.0             & P, F        & S0\\
    L104  & 3.0             & P, F        & L1, L2, L4, H5, H6, H7\\
    S104  & 1.5             & P, F        & S3, S8\\
    T104  & 3.0             & A--F        & Exam\\
    \midrule
    Total & 7.5             & A--F        & (Determined by exam)\\
    \bottomrule
  \end{tabular}
  \caption{%
    Table summarizing course modules and their mapping to LADOK\@.
    P means pass, F means fail.
    A--E are also passing grades, where A is the best.
  }\label{LADOKTable}
\end{table}

\end{frame}

The written exam will be graded A--E for passing grades, F or Fx for failing 
grades.
You will receive an Fx if you are very close to passing.
In this case you may complement your written exam with an oral exam.
If you do not take this chance you must retake the exam the next 
time it is given.
The grade of the exam will also be the grade of the course total.

\subsection{Handed-in assignments}
\mode<presentation>{%
  \begin{frame}
    \begin{itemize}
      \item Must be in \enquote{passable} condition.
      \item Otherwise rejection without comment.

        \pause

      \item No plagiarism accepted.
      \item When working in group: everyone accountable.
    \end{itemize}
  \end{frame}
}

In general, all hand-ins in the course must be in a \enquote{passable} 
condition; \ie they must be well-written, grammatically correct and without 
spelling errors, have citations and references according to~\cite{IEEEcitation} 
(see also~\cite{PurdueCitation} for a tutorial), and finally fulfil all 
requirements from the assignment instruction.
If you hand something in which is not in this condition, you will receive an 
F without further comment.

All material handed-in must be created by yourself, or, in the case of group 
assignments, created by you or one of the group members.
When you refer to or quote other texts, then you must provide a correct list of 
references and, in the case of quotations, the quoted text must be clearly 
marked as quoted.
If any part of the document is plagiarized you risk being suspended from study 
for a predetermined time, not exceeding six months, due to disciplinary 
offence.
If it is a group assignment, all group members will be held accountable for 
disciplinary offence unless it is clearly marked in the work who is responsible 
for the part containing the plagiarism.

If cooperation takes place without the assignment instruction explicitly 
allowing this, this will be regarded as a disciplinary offence with the risk of
being suspended for a predetermined time, not exceeding six months.
Unless otherwise stated, all assignments are to be done individually.


\subsection{\enquote{What if I'm not done in time?}}%
\label{sec:late}
\mode<presentation>{%
  \begin{frame}
    \begin{itemize}
      \item You have three chances for grading per year.
      \item These are marked in the schedule.
      \item Thus there will be three deadlines per assignment until the next 
        time the course is given.
    \end{itemize}
  \end{frame}

  \begin{frame}
    \begin{itemize}
      \item No tutoring is planned after the course.
      \item If you want to ensure tutoring, it's during the course.
    \end{itemize}
  \end{frame}

  \begin{frame}
    \begin{alertblock}{If you predict you will not finish on time}
      \begin{itemize}
        \item Within three weeks of course start, deregister from the course.
        \item This allows you to reregister next time the course is given.

          \pause

        \item You must reregister to get access to the course the following 
          year.
        \item If you haven't cancelled, you'll be last in the queue.
      \end{itemize}
    \end{alertblock}
  \end{frame}
}

The deadlines on this course are of great importance, make sure to keep these!
You must have completed the introductory assignment within its deadline.
If you do not do this you will be deregistered from the course and your place 
will be open to other students.

For seminars and presentations there will be three sessions during the course 
of a year, if you cannot make it to any of those you will have to return the 
next time the course is given; \ie up to a year later.
All of these sessions will be in the course schedule (in the Student Portal).
If you miss a deadline for the preparation for a seminar session, then you have 
to go for the next seminar even if the first seminar has not passed yet.

Written assignments are graded once during the course, most often shortly after 
the deadline of the assignment.
After the course you are offered two more attempts within a year.
In total you have three chances for having your assignments graded over the 
period of a year.
After that you should come back the next time the course is given.

No tutoring is planned after the end of the course, \ie after the last 
tutoring session scheduled in the course schedule.
If you are not done with your assignments during the course and want to be 
guaranteed tutoring you have to reregister for the next time the course is 
given.
Reregistration is a lower priority class of applicants for a course, all 
students applying for the course the first time have higher priority --- this 
includes reserves too.

%If you by the end of the course have a majority of the assignments left undone 
%you will have to reregister for the course the next time it is given.
%Whether you have completed the majority of the assignments or not is up to the 
%teacher to decide.
%Talk to the teacher to see if you have to reregister or can just hand in the 
%missing assignments.

Thus, if you feel that you will not be done with the course on time, it is 
better to stop the course at an early stage.
If you register a break within three weeks of the course start, you will be in 
the higher priority class of applicants the next time you apply for the course.
You can register such a break yourself in the Student Portal.




\printbibliography{}
