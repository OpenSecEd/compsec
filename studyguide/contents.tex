\title{%
  A study guide for a first course in\\
  Computer Security
}
\author{%
  Daniel Bosk
}
\institute{%
  Department of Information and Communication Systems\\
  Mid Sweden University, SE-851\,70 Sundsvall
}

\maketitle


\section{Scope and aims}
\label{sec:aim}
The course aims towards a good understanding for the requirements of a secure 
computer system.
Problems such as authentication and access control; software security, such as 
buffer overflows; as well as operating system, library and application security 
mechanisms are treated in the course.

More specifically, after taking this course you should be able to:
\begin{itemize}
  \item apply different cryptosystems and explain how these work,
  \item analyse the problems of authentication, access control and different 
  solutions,
  \item explain how some common attacks on software works,
  \item analyse different operating system security mechanisms,
  \item analyse the functionality of different types of malware,
  \item explain different malware protection mechanisms,
  \item evaluate strengths and weaknesses of hardware-based security and 
    full-disk encryption, as well as
  \item value and argue about different ethical aspects of computer security, 
    e.g.\ surveillance.
\end{itemize}


\section{Overview of structure and content}
\label{CourseOutline}
% - Basic cryptography: block ciphers, asymmetric cryptosystems, digital 
%   signatures, hash functions.
% - User authentication, access control and intrusion detection.
% - Software security: buffer overflow, handling input, interaction between 
%   programs.
% - Operating system security.
% - Malware.
% - Trusted computing, hardware based security and whole disk encryption.
% - Ethical aspects of computer security.
The course covers
applied cryptography used in computer security, e.g.~uses of cryptography for 
code obfuscation or digital rights management;
authentication mechanisms, access control, and intrusion detection;
software security, e.g.~buffer overruns and interaction between programs;
some security mechanisms provided by operating system and hardware;
and malicious software and how these utilise the above weaknesses.
Finally we discuss some ethical implications for computer engineers.

\subsection{Teaching}

The main course literature is \citetitle{Gollmann2011cs} by 
\citeauthor{Gollmann2011cs}~\cite{Gollmann2011cs}.
This is complemented by \citetitle{Anderson2008sea} by 
\citeauthor{Anderson2008sea}~\cite{Anderson2008sea}.
The course is taught using lectures, individual laboratory assignments, 
workshops (\enquote{hackathon labs}), seminars, and finally a written exam.
You can find a more detailed timetable, containing lab sessions etc., in the 
following subsection.
All assignments are numbered consecutively prefixed with an \enquote{L} for 
laboratory assignments, \enquote{H} for hackathons and \enquote{S} for seminar 
assignments.
For general information about the examination of these and deadlines, see 
\cref{Assessment}.
For detailed information, please see the instructions found in the course 
platform.

\subsection{Schedule}
\label{Schedule}

To make your reading of the course easier, you are presented with a suggested 
schedule in this section.
You are free to follow this schedule or any schedule you make for yourself, but 
the deadlines, laboratory sessions, and lectures will follow this schedule.
You will find a short summary of the schedule in \cref{TimeTable}.
The detailed reading instructions for each item in the schedule can be found in 
the following sections.

\begin{table}
	\centering
  \begin{tabular}{lp{9cm}}
    \toprule
    \textbf{Course week}	& \textbf{Work} \\
    \midrule
    1
    & Lecture: Course start/Foundations of security\\
    & Lecture: Security usability\\
    \midrule
    2
    & Individual study\\
    \midrule
    3
    & Lecture: Information theory\\
    & Lecture: Cryptography, part I\\
    & Lecture: Cryptography, part II\\
    & Lab: L0 (spuriouslab)\\
    \midrule
    4
    & Lecture: Identification and authentication, part I\\
    & Lecture: Identification and authentication, part II\\
    & Seminar: S2 (pwdpolicies)\\
    & Lab: L0 (spuriouslab), L1 (pwdguess), L3 (pricomlab)\\
    \midrule
    5
    & Lecture: Access control\\
    & Lecture: Reference monitors\\
    %& Lecture: UNIX and Windows Security\\
    %& Lecture: Database Security\\
    & Lecture: Accountability\\
    & Lab: L1 (pwdguess), L3 (pricomlab), L4 (tools)\\
    \midrule
    6
    & Lecture on Trusted computing\\
    & Lecture on Side-channels\\
    & Lecture on Software security\\
    & Lab: L1 (pwdguess), L3 (pricomlab), L4 (tools)\\
    \midrule
    7
    & Hackathon: H5 (drmlab)\\
    & Hackathon: H6 (stacksmash)\\
    & Lab: L1 (pwdguess), L3 (pricomlab), L4 (tools)\\
    \midrule
    8
    & Hackathon: H7 (malwarelab)\\
    \midrule
    9
    & Presentation: L3 (tools)\\
    & Seminar: S8 (ethics)\\
    \midrule
    10
    & Exam: course exam\\
    & Lab: last call for all labs\\
    & Seminar: second call on all seminars, incl L4 presentation\\
    \midrule
    +3 months
    & Exam: re-exam\\
    & Seminar: final call on all seminars, incl L4 presentation\\
    \midrule
    +6 months
    & Exam: last re-exam until next course\\
    \bottomrule
  \end{tabular}
  \caption{%
    A summary of the parts of the course and when they will (or should) be 
    covered.
    The table is adapted to taking this course on half-time study rate, i.e.\ 
    over ten weeks.
  }\label{TimeTable}
\end{table}


\section{Course content}

This section summarizes the material covered by the lectures and assignments, 
i.e.\ what you should read for each of them.
It is divided by topics and ordered according to progression of the course.

\subsection{Foundations of security}
\input{foundations-lit.tex}

\subsection{Information theory}
\input{infotheory-lit.tex}

\subsection{Cryptography}
\input{crypto-lit.tex}

\subsection{L0 May the (brute) force be with you}
\input{spuriouslab-lit.tex}

\subsection{Identification and authentication}
\input{auth-lit.tex}

\subsection{Security usability}
\input{usability-lit.tex}

\subsection{L1 Password cracking and social engineering}
\input{pwdguess-lit.tex}

\subsection{S2 Password policies}
\input{pwdpolicies-lit.tex}

\subsection{L3 Private communication}
\input{pricomlab-lit.tex}

\subsection{Access control}
\input{accessctrl-lit.tex}

\subsection{Reference monitors}
\input{refmon-lit.tex}

%\subsection{UNIX security}
%\input{unix-lit.tex}

%\subsection{Windows security}
%\input{windows-lit.tex}

\subsection{Accountability}
\input{accountability-lit.tex}

\subsection{L4 Tools of the trade}
\input{tools-lit.tex}

\subsection{Trusted computing}
\input{trustcomp-lit.tex}

\subsection{H5 Digital rights management}
\input{drm-lit.tex}

\subsection{Side-channels}
\input{sidechannels-lit.tex}

\subsection{Software security}
\input{software-lit.tex}

\subsection{H6 Smashing the stack}
\input{stacksmash-lit.tex}

\subsection{H7 Malicous software}
\input{malware-lit.tex}

\subsection{S8 The computer engineer's code of ethics}
\input{ethics-lit.tex}

\subsection{Final exam}

The final exam will assess how well you have achieved the intended learning 
outcomes of the course.
Hence, it covers all the content given above.


\section{Assessment}
\label{Assessment}

This section explains how the course modules are graded and mapped to LADOK\@.
\Cref{LADOKTable} visualizes the relations between modules, credits, grades and 
LADOK\@.

\begin{table}
  \centering
  \caption{%
    Table summarizing course modules and their mapping to LADOK\@.
    P means pass, F means fail.
    A--E are also passing grades, where A is the best.
  }\label{LADOKTable}
  \begin{tabular}{rrll}
    \toprule
    LADOK & Credits (ECTS)  & Grade       & Course Assignments\\
    \midrule
    I104  & 0.0             & P, F        & L0\\
    L104  & 3.0             & P, F        & L1, L3, L4, H5, H6\\
    S104  & 1.5             & P, F        & S2, S7\\
    T104  & 3.0             & A--F        & Exam\\
    \midrule
    Total & 7.5             & A--F        & (Determined by exam)\\
    \bottomrule
  \end{tabular}
\end{table}

The written exam will be graded A--E for passing grades, F or Fx for failing 
grades.
You will receive an Fx if you are very close to passing.
In this case you may complement your written exam with an oral exam within 
a week from receiving the result.
If you do not take this chance within a week you must retake the exam the next 
time it is given.
The grade of the exam will also be the grade of the course total.

\subsection{Handed-in assignments}

In general, all hand-ins in the course must be in a \enquote{passable} 
condition; i.e.\ they must be well-written, grammatically correct and without 
spelling errors, have citations and references according to~\cite{IEEEcitation} 
(see also~\cite{PurdueCitation} for a tutorial), and finally fulfil all 
requirements from the assignment instruction.
If you hand something in which is not in this condition, you will receive an 
F with the comment \enquote{incomplete}.

All handed-in material must be created by yourself, or, in the case of group 
assignments, created by you or one of the group members.
When you refer to or quote other texts, then you must provide a correct list of 
references and, in the case of quotations, the quoted text must be clearly 
marked as quoted.
If any part of the document is plagiarized you risk being suspended from study  
for a predetermined time, not exceeding six months, due to disciplinary 
offence.
If it is a group assignment, all group members will be held accountable for 
disciplinary offence unless it is clearly marked in the work who is responsible 
for the part containing the plagiarism.

If cooperation takes place without the assignment instruction explicitly 
allowing this, this will be regarded as a disciplinary offence with the risk of 
being suspended for a predetermined time, not exceeding six months.
Unless otherwise stated, all assignments are to be done individually.

\subsection{\enquote{What if I'm not done in time?}}
\label{WhatIfImLate}
\mode<presentation>{%
  \begin{frame}
    \begin{itemize}
      \item You have three chances for grading per year.
      \item These are marked in the schedule.
      \item Thus there will be three deadlines per assignment until the next 
        time the course is given.
    \end{itemize}
  \end{frame}

  \begin{frame}
    \begin{itemize}
      \item No tutoring is planned after the course.
      \item If you want to ensure tutoring, it's during the course.
    \end{itemize}
  \end{frame}

  \begin{frame}
    \begin{alertblock}{If you predict you will not finish on time}
      \begin{itemize}
        \item Within three weeks of course start, deregister from the course.
        \item This allows you to reregister next time the course is given.

          \pause

        \item You must reregister to get access to the course the following 
          year.
        \item If you haven't cancelled, you'll be last in the queue.
      \end{itemize}
    \end{alertblock}
  \end{frame}
}

The deadlines on this course are of great importance, make sure to keep these!
You must have completed the introductory assignment within its deadline.
If you do not do this you will be deregistered from the course and your place 
will be open to other students.

For seminars and presentations there will be three sessions during the course 
of a year, if you cannot make it to any of those you will have to return the 
next time the course is given; \ie up to a year later.
All of these sessions will be in the course schedule (in the Student Portal).
If you miss a deadline for the preparation for a seminar session, then you have 
to go for the next seminar even if the first seminar has not passed yet.

Written assignments are graded once during the course, most often shortly after 
the deadline of the assignment.
After the course you are offered two more attempts within a year.
In total you have three chances for having your assignments graded over the 
period of a year.
After that you should come back the next time the course is given.

No tutoring is planned after the end of the course, \ie after the last 
tutoring session scheduled in the course schedule.
If you are not done with your assignments during the course and want to be 
guaranteed tutoring you have to reregister for the next time the course is 
given.
Reregistration is a lower priority class of applicants for a course, all 
students applying for the course the first time have higher priority --- this 
includes reserves too.

%If you by the end of the course have a majority of the assignments left undone 
%you will have to reregister for the course the next time it is given.
%Whether you have completed the majority of the assignments or not is up to the 
%teacher to decide.
%Talk to the teacher to see if you have to reregister or can just hand in the 
%missing assignments.

Thus, if you feel that you will not be done with the course on time, it is 
better to stop the course at an early stage.
If you register a break within three weeks of the course start, you will be in 
the higher priority class of applicants the next time you apply for the course.
You can register such a break yourself in the Student Portal.




\printbibliography{}
