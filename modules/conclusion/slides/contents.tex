\title{%
  Conclusions of the course
}
\author{Daniel Bosk}
\institute{%
  MIUN IST
}

\mode<article>{\maketitle}
\mode<presentation>{%
  \begin{frame}
    \maketitle
  \end{frame}
}

\mode*

\begin{abstract}
  \mode*

% What's the problem?
% Why is it a problem? Research gap left by other approaches?
% Why is it important? Why care?
% What's the approach? How to solve the problem?
% What's the findings? How was it evaluated, what are the results, limitations, 
% what remains to be done?

% XXX Summary
\emph{Summary:}
In this learning session we will cover the foundations of security.
By this we mean what security is all about, \eg what types of properties we are 
interested in and what we want to achieve in our security work.
We will also introduce the scientific method and, particularly, how this can be 
applied in the area of security to create new knowledge.

Finally, we will cover usability and how users' weaknesses affects security.
There are many ways to attack systems through their human operators.
We must consider usability when designing secure systems.

% XXX Motivation and intended learning outcomes
\emph{Intended learning outcomes:}
After this session you should be able:
\begin{itemize}
  \item to \emph{understand} the what security is generally about.
  \item to \emph{differentiate} which types of scientific methods are 
    appropriate to answer a given question.
  \item to \emph{adopt} an adversarial thinking for situtions involving humans.
  \item to \emph{incorporate} basic psychology in the design of a system to 
    increase its security.
\end{itemize}

% XXX Prerequisites
%\emph{Prerequisites:}
%\dots

% XXX Reading material
\emph{Reading:}
You should read Gollmann's chapter on \enquote{Foundations of Computer 
  Security}~\cite[Chap.\ 3]{Gollmann2011cs}.
There he attempts at a definition of Computer Security and related terms, \eg 
confidentiality, integrity, and availability, which we need for our treatment of 
the topic.
Anderson also covers this in Chapter 1 of~\cite{Anderson2008sea}.
He also treats a wider area than just \emph{computer} security, which is good 
for us, he covers many aspects of security in different examples.

For the introduction to the scientific method you should read 
\citetitle{HowToDesignSecurityExperiments}~\cite{HowToDesignSecurityExperiments}.
This paper discusses the scientific method of (parts of) the security field.
For a more in-depth reflection on the state of security as a scientific pursuit, 
we recommend
\citetitle{SecurityAsAScience}~\cite{SecurityAsAScience}.

\Citeauthor{Anderson2008sea} gives a short summary of the psychology of users, 
their strengths and weaknesses, and how usability affects security in Chapter 2 
\enquote{Usability and Psychology} of 
\citetitle{Anderson2008sea}~\cite{Anderson2008sea}.

\end{abstract}


%\section{Summary of the course}
%
%\subsection{Foundations}
%
%\paragraph{What is security?}
%
%\emph{Summary:}
%In this learning session we will cover the foundations of security.
%By this we mean what security is all about, \eg what types of properties we are 
%interested in and what we want to achieve in our security work.
%
%\emph{Intended learning outcomes:}
%After this session you should be able:
%\begin{itemize}
%  \item to \emph{understand} the what security is generally about.
%\end{itemize}
%
%\emph{Reading:}
%You should read Gollmann's chapter on \enquote{Foundations of Computer 
%  Security}~\cite[Chap.\ 3]{Gollmann2011cs}.
%There he attempts at a definition of Computer Security and related terms, \eg 
%confidentiality, integrity, and availability, which we need for our treatment of 
%the topic.
%Anderson also covers this in Chapter 1 of~\cite{Anderson2008sea}.
%He also treats a wider area than just \emph{computer} security, which is good 
%for us, he covers many aspects of security in different examples.
%
%\paragraph{The scientific method}
%
%\emph{Summary:}
%In this learning session we will give an introduction to the scientific method 
%and particularly how this can be applied in the area of security.
%
%\emph{Intended learning outcomes:}
%After this session you should be able:
%\begin{itemize}
%  \item to \emph{differentiate} which types of scientific methods are 
%    appropriate to answer a given question.
%\end{itemize}
%
%\emph{Reading:}
%You should read 
%\citetitle{HowToDesignSecurityExperiments}~\cite{HowToDesignSecurityExperiments}.
%This paper discusses the scientific method of (parts of) the security field.
%For a more in-depth reflection on the state of security as a scientific pursuit, 
%we recommend
%\citetitle{SecurityAsAScience}~\cite{SecurityAsAScience}.
%
%\paragraph{Usability}
%
%One important aspect of security, which traditionally is forgotten, is the 
%users' weaknesses.
%The psychology of the human mind is therefore an important subject to discuss 
%in the context of security.
%And consequently, we must adapt our systems to those limitations.
%How the users function and how to adapt systems to their limitations is at the 
%centre of the usability area.
%
%Anderson gives a short summary of the psychology of users, their strengths and 
%weaknesses, in Chapter 2 ``Usability and Psychology'' in 
%\cite{Anderson2008sea}.
%Also treated here is the ever-recurring problem of password policies.
%The material covering this area is the article 
%\citetitle{Komanduri2011opa}~\cite{Komanduri2011opa} and its follow-up article 
%\citetitle{Komanduri2014can}~\cite{Komanduri2014can}.
%Base on the abstracts for the modules.


\section{The exam}

Go through the rules of the exam.
Go through a few questions.



%%% REFERENCES %%%

\begin{frame}[allowframebreaks]
  \printbibliography
\end{frame}
