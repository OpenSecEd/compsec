\paragraph{Basic information theory}

The area of Information Theory was founded in 1948 by Claude Shannon.
It is a mathematical theory to reason about how much information is contained 
in certain data.
Equivalently, it is also a measure of uncertainty in information, and has thus 
plenty of application in security and cryptography.
This learning session covers the basic concept, Shannon entropy, and some 
applications to security and privacy.

After the session you should be able
\begin{itemize}
  \item to \emph{apply} Shannon entropy in basic situations related to security 
    and privacy.
\end{itemize}

The concept of Shannon entropy, the main part of information theory, is treated 
in a few short texts:
\citetitle{Eckersley2010apo}~\cite{Eckersley2010apo} and
\citetitle{Ueltschi2013se}~\cite{Ueltschi2013se}.
You should read on the use of entropy to estimate identifiability:
\citetitle{Eckersley2010hui}~\cite{Eckersley2010hui}.

\paragraph{A high-level overview of crypto}

Cryptography has a central role in security.
To fully understand how many security mechanisms can be implemented we need 
cryptography.
For this reason, we also need higher-level knowledge about what can be achieved 
with cryptography to not limit our thoughts about possible solutions.
This learning session is intended to give a high-level overview of 
cryptography: \ac{SKE}, \ac{PKE}, digital signatures, \ac{ZKP} and \ac{MPC}.
In particular, the \acp{ILO} are that you should be able to
\begin{itemize}
  \item \emph{understand} what properties can be achieved with cryptography.
  \item \emph{analyse} a situation and \emph{suggest} what cryptographic 
    properties are desirable.
\end{itemize}

The basics are covered by
Chapter 5 in Anderson's \citetitle{Anderson2008sea}~\cite{Anderson2008sea} and
Chapter 14 in Gollmann's \citetitle{Gollmann2011cs}~\cite{Gollmann2011cs}.
(To practice your understanding of these mechanisms it is recommended to do 
exercises 14.2, 14.3 and 14.7 in~\cite{Gollmann2011cs}.)
%We will also cover some topics from 
%\citetitle{KatzLindell-v1}~\cite{KatzLindell-v1} and 
%\citetitle{GoldreichFOC-1}~\cite{GoldreichFOC-1}.
For the remaining topics, however, we refer to the \citetitle{EOCS}~\cite{EOCS} 
(and cited papers and books).
