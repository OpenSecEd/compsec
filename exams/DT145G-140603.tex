% $Id$
% Author:  Daniel Bosk <daniel.bosk@miun.se>
\documentclass[svv,addpoints]{miunexam}
\usepackage[utf8]{inputenc}
\usepackage[T1]{fontenc}
\usepackage[swedish,english]{babel}
\usepackage[hyphens]{url}
\usepackage{hyperref}
\usepackage{color}
\usepackage{prettyref,varioref}
\usepackage{subfigure}
\usepackage{amsmath,amssymb}
\usepackage{listings}
\usepackage[natbib,style=alphabetic,maxbibnames=99,sorting=none,backend=bibtex8]{biblatex}
\addbibresource{literature.bib}
\usepackage[varioref,prettyref,listings]{miunmisc}

\examtype{Final exam}
\courseid{DT145G}
\course{Computer Security}
\date{2014-06-03}
%\printanswers
\author{%
  Daniel Bosk\\
  {\small\texttt{\href{mailto:daniel.bosk@miun.se}{daniel.bosk@miun.se}}}\\
  {\small\textit{Phone:} 060\,-\,14\,8709}
}

\DeclareMathOperator{\hmac}{HMAC}
\DeclareMathOperator{\xor}{\oplus}
\DeclareMathOperator{\concat}{||}

\begin{document}
\maketitle
\thispagestyle{foot}

\section*{Instructions}
\label{sec:Instructions}
Carefully read the questions before you start answering them.
Note the time limit of the exam and plan your answers accordingly.
Only answer the question, do not write about subjects remotely related to the
question.

Write your answers on separate sheets, not on the exam paper.
Only write on one side of the sheets.
Start each question on a new sheet.
Do not forget to \emph{motivate your answers.}

Make sure you write your answers clearly, if I cannot read an answer the answer
will be awarded no points -- even if the answer is correct.
The questions are \emph{not} sorted by difficulty.

\begin{description}
  \item[Time] 5 hours.
  \item[Aids] Dictionary.
  \item[Maximum points] \numpoints
  \item[Questions] \numquestions
\end{description}

%\subsection*{Bonus points}
%\noindent
%You must get an E or higher, to get the bonus points added to your final grade.
%Bonus points will be added to this exam and the first re-exam.

\subsection*{Preliminary grades}
The following grading criteria applies:
E \(\geq 50\%\),
D \(\geq 60\%\),
C \(\geq 70\%\), 
B \(\geq 80\%\),
A \(\geq 90\%\).%;
%with no question awarded zero points.

%To pass this exam you need to have at least an average of one (1) point per 
%question, and no question may be awarded zero points.
%The averages for the grades can be summarised as follows:
%E \(\geq 1\),
%D \(\geq 1.5\),
%C \(\geq 2\),
%B \(\geq 2.5\),
%A \(\geq 3\),
%with no question awarded zero points.


\clearpage
\section*{Questions}
The questions are given below.
They are not given in any particular order.

\begin{questions}

  \question\label{q:terminology}
  Explain the following terms:
  \begin{parts}
    \part[1] Confidentiality
    \part[1] Integrity
    \part[1] Availability
    \part[1] Accountability
    \part[1] Non-Repudiation
  \end{parts}
  \begin{solution}
    See \cite{Gollmann2011cs} and \cite{Anderson2008sea} for definitions.
  \end{solution}


  \question\label{q:non-repudiation}
  A user has set the access control list of a file in such a way that only she 
  has access to this file.
  \begin{parts}
    \part[3] If you would like to read this file you can, explain how you would 
    do and why it works (also what assumptions you need).

    \part[3] Suggest how the user must do to guarantee that no one but herself 
    can read this file, also motivate why this protection works.
  \end{parts}
  \begin{solution}
    The way she's securing her file is by using access control mechanisms in 
    the operating system (OS).

    Assuming we have physical access to the computer, then we can just read the 
    raw data from the disk.
    This can be accomplished by either booting our own OS on her computer, or 
    by removing the disk.

    If we don't have physical access we can always try to bypass the access 
    control mechanisms in other ways, e.g.\ by gaining privileges in the system 
    or seeing if the OS has failed to protect reading from the raw disk (i.e.\ 
    not using the file system).

    The most obvious way to have system independent security for this file is 
    to encrypt it.
    This way no one can read it unless they have access to the key, and this is 
    true no matter if you change the environment.
    (Of course, if the system is untrusted someone can get to the key that way, 
    but that's outside the scope of this question.)
  \end{solution}


  \question\label{q:infogain}
  The University password composition policy is as follows\footnote{%
    It's slightly more advanced than this, but it's simplified here for reasons 
    of convenience.
    Also, the effect might actually be the same anyway, depending on if users 
    actually use the extension or not.
  }:
  The password must be at least eight (8) characters long, further, the first 
  eight characters must contain two uppercase and two lowercase letters as well 
  as two numbers.

  \begin{parts}
    \part[3] Compute the information gained by an attacker from knowing this 
    policy.

    \part[3] Compare this to the information gained if the policy would have 
    been ``the password must be at least 16 characters long''.
  \end{parts}
  \begin{solution}
    We assume uniform randomness, this yields the highest entropy.
    We know that users don't use uniform randomness, so the actual entropy will 
    be lower and thus this will be the upper bound.

    The entropy per lower-case letter in the alphabet is approximately 5 bits, 
    the same for the upper-case letters, it's approximately 3 bits for the 
    numbers, it's approximately 6 bits for all letters and numbers together 
    (\(64 = 2^6\)).
    Hence, a password of length 8 would yield \(6\times 8 = 48\) bits of 
    entropy.
    However, we know from \cite{Komanduri2011opa} that this is more around 30 
    bits.

    If the attacker learns the password composition policy, this reduces 
    entropy by:
    2 bits for lower-case,
    2 bits for upper-case,
    6 bits for numbers.
    I.e.\ a total of 10 bits reduction of entropy.
    This leaves us with a maximal entropy of 38 bits.
    We know from \cite{Komanduri2011opa} that this is closer to 34 bits of 
    entropy.

    If we look at passwords of length 16 we get a maximum entropy of \(6\times 
    16\) bits.
    However, we know from \cite{Komanduri2011opa} that this is closer to 45 
    bits.
  \end{solution}


  \question\label{q:drm}
  The DRM manager in a company claims to have devised a secure DRM for the 
  company's software product.

  It works as follow:
  To use the software the user has to enter a authentication code provided with 
  the software.
  If valid the software will connect to the servers and use some functionality 
  stored there.

  For the connection the software computes a MAC of itself (the executable 
  file) using HMAC-SHA256 and a secret key \(k\) shared with the company 
  servers, this is to prevent modification attempts of the executable file.
  It then connects to one of these servers, it verifies the server's 
  certificate to establish a secure connection, then sends the MAC which is 
  verified by the server, the server verifies and performs the software's 
  request if the MAC is valid.

  There are several flaws in this DRM scheme.
  \begin{parts}
    \part[3] Explain what can be done to work around this DRM.
    (A solution which doesn't need a copy of a valid authentication code is of 
    course better than a solution which does.)

    \part[3] Suggest some improvements of this DRM and motivate what 
    vulnerabilities each improvement removes.
    Also motivate whether your improved version is totally secure or not.
  \end{parts}
  \begin{solution}
    The authentication step is a mechanism only implemented in the software 
    itself, hence, it can be bypassed since we control the software -- we just 
    remove that code.

    This is supposed to be countered by the MAC in the connection step, 
    however, this mechanism is also implemented in the software itself.
    The secret key \(k\) must be available to the software, and hence to us 
    too.
    We can compute the MAC using \(k\) and the unmodified version of the 
    software, then we rewrite this part of the software to only send this 
    hard-coded MAC and not do any new computations.

    This way we can still use the remote functionality without ever seeing 
    a valid authentication code.

    The obvious way to improve security here is to verify the authentication 
    code on the server side.
    This way we need a valid authentication code to use the software, hence we 
    can have some liability for those sharing their authentication code -- if 
    these are connected to some person.

    There is still the possibility to reverse engineer the part done by the 
    server, then we could use the software without authentication again.
    However, this probably requires enough effort to rewrite the entire program 
    anyway, so it's not likely anyone would do it.

    There are of course other ways to secure this too, this is just an example, 
    and research is ongoing in this area as well.
  \end{solution}


  \question\label{q:stacksmash}
  Look at the C code in \prettyref{lst:overrun}.
  \begin{parts}
    \part[3] Identify all vulnerabilities in that code and motivate by stating 
    how they can be exploited.

    \part[3] Suggest improvements to remedy these vulnerabilities, you must 
    motivate why they work.
  \end{parts}
  \begin{solution}
    There is a possibility for stack smashing in \code{get_some_input} and 
    \code{main}.
    Check the boundaries of buffers, never let scanf(3) read more data than the 
    buffer can hold.

    There is an integer overflow in \code{make_full_name}.
    A size can never be negative, hence don't use signed integers.
    Use the \code{size_t} datatype, which is defined as unsigned.
    This way we can only overflow using twice the data.
    Another way is to check if we expect an overflow to occur, e.g.\ if both 
    sizes are larger than half of the maximum value.
  \end{solution}

  \begin{src}[float,caption={Some vulnerable C code.},label={lst:overrun}]
#include <stdio.h>

int
get_some_input( void )
{
  char buffer[128];

  printf( "Please enter the key: " );
  scanf( "%s", buffer );

  /* process input */

  return 0;
}

void
make_full_name( char *dst, int dstlen,
                const char *src, int srclen,
                int maxsize )
{
  if ( dstlen + srclen + 1 >= maxsize )
    return -1;

  strncat( dst, " ", 1 );
  return strncat( dst, src, srclen );
}

int
main( int argc, char **argv )
{
  char first[256];
  char last[256];

  printf( "Please enter your first name: " );
  scanf( "%s", first );
  printf( "Please enter you last name: " );
  scanf( "%s", last );

  make_full_name( first, strlen( first ),
                  last, strlen( last ), 256 );

  if ( get_some_input() < 0 )
    return -1;

  return 0;
}
  \end{src}

\end{questions}


\printbibliography
\end{document}

