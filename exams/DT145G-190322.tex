% Author:  Daniel Bosk <daniel.bosk@miun.se>
\documentclass[svv,addpoints]{miunexam}
\usepackage[utf8]{inputenc}
\usepackage[T1]{fontenc}
\usepackage[swedish,english]{babel}
\usepackage[hyphens]{url}
\usepackage{hyperref}
\usepackage{color}
\usepackage{prettyref,varioref}
\usepackage{subfigure}
\usepackage{amsmath,amssymb}
\usepackage{listings}
\usepackage{authblk}

\usepackage{csquotes}
\MakeBlockQuote{<}{|}{>}
\EnableQuotes

\usepackage[natbib,style=alphabetic,maxbibnames=99]{biblatex}
\addbibresource{literature.bib}

\printanswers

\examtype{Final exam}
\courseid{DT145G}
\course{Computer Security}
\date{2019-03-22}
\author{%
  Daniel Bosk
}
\affil{%
  Department of Information Systems and Technology,\\
  Mid Sweden University, SE-851\,70 Sundsvall\\
  Email: \href{mailto:daniel.bosk@miun.se}{daniel.bosk@miun.se}\\
  Phone: 010-142\,8709
}

\DeclareMathOperator{\hmac}{HMAC}
\DeclareMathOperator{\xor}{\oplus}
\DeclareMathOperator{\concat}{||}

\begin{document}
\maketitle
\thispagestyle{foot}

\section*{Instructions}
\label{sec:Instructions}
Carefully read the questions before you start answering them.
Note the time limit of the exam and plan your answers accordingly.
Only answer the question, do not write about subjects remotely related to the
question.

Write your answers on separate sheets, not on the exam paper.
Only write on one side of the sheets.
Start each question on a new sheet.
Do not forget to \emph{motivate your answers.}

Make sure you write your answers clearly, if I cannot read an answer the answer
will be awarded no points---even if the answer is correct.
The questions are \emph{not} sorted by difficulty.

\begin{description}
  \item[Time] 5 hours.
  \item[Aids] Dictionary.
  \item[Questions] \numquestions
\end{description}

\subsection*{Preliminary grades}

Each question can be awarded up to three points: one point for E, two points 
for C and three points for A.
To get an E, you must get at least one point on each question --- i.e.\ no 
question must be awarded zero points.
For a C, you must get two points on at least half of the questions.
For an A, you must get three points on at least half of the questions.


\clearpage
\section*{Questions}
The questions are given below.
They are not given in any particular order.

\begin{questions}
  \question[3]
% tags: passwd:E:C:A
% tags: infotheory:E:C:A
\enquote{Well, every respectable website requires at least eight characters, 
with lower and upper case, numbers and special characters}, your boss says, 
\enquote{so we can have it too}.

What would you like to say in this conversation?
(First, consider the alternatives to passwords.
Second, how would you do passwords properly?)

\begin{solution}
There are better alternatives to passwords.
E.g.\ we can use cryptographic techniques instead, BankID is a good example.
Then we would reduce the problem of breaches as well, no problem leaking public 
keys since they're public.

If we are to use passwords: The last decades' research in user authentication 
says that such a policy yields bad security.
It forces users to select easy to guess passwords and incentivizes password 
reuse while more secure passwords are disqualified according to the policy.
A better policy is to have at least 12 characters as the only requirement.
Also, no requirement of updating the password at regular intervals --- only if a 
breach has occurred.
\end{solution}
\question[3]
% tags: sidechannels:E:C:A
\begin{parts}
  \part Give an example of a covert channel and
  \begin{solution}
    A server is anonymous (e.g.\ a Tor hidden service), i.e.\ you may access 
    the server but not know its location.
    Part of the server's service is giving the time.
    It has been shown that the variations in the system clock depend on the 
    ambient temperature.
    This means that by studying how the time on the server varies over day and 
    night and over the seasons, we can eventually figure out the ambient 
    temperature.
    From the ambient temperature we can later deduce the geographical location 
    of the server.
  \end{solution}

  \part what we can do about it.
  \begin{solution}
    We can lower the resolution in the time-stamps the server gives, e.g.\ by 
    not giving seconds.
    This lowers the bandwidth of the covert channel, perhaps so that the attack 
    is infeasible.
    We could also sync the servers clock more often, e.g.\ by using the Network 
    Time Protocol.
    However, the only way to prevent it is by not revealing the time of the 
    server's system clock.
  \end{solution}
\end{parts}



\question[3]
% tags: accountability:E:C:A
What is the purpose of separation of duty?
Explain and illustrate your explanation with an example.

\begin{solution}
  The purpose of separation of duty is to make it more difficult for 
  a malicious entity to subvert the system.
  E.g.\ a systems developer at the bank should not be able to add a back-door 
  for himself so that he later can steal money without anyone noticing.
  If he is only allowed to write the code, but not to certify its correct 
  function --- then there is a higher chance that he is caught before the 
  system is launched.
  So to subvert the system a malicious actor must act together with others.
\end{solution}


\question[3]
  % tags: usability:crypto:E:C:A
  There are numerous alternatives available today for end-to-end secure 
  communication\footnote{%
    Normally this is referred to as end-to-end \emph{encrypted} communication, 
    but we have integrity in addition to confidentiality.
  }, e.g.\ the apps Signal, WhatsApp and Telegram for instant messaging, media 
  messaging and video calls; PGP and S/MIME for email.
  They are all based on a combination of public-key and shared-key cryptography.

  Discuss the security and usability challenges facing end-to-end secure 
  communication.

  \begin{solution}
    The challenge for these systems is to provide a design which aligns the 
    security with how users work.

    The easy part is to provide a tool for securing the communication, i.e.\ 
    encrypting and providing integrity checks.

    The hard part is to do proper key management, specifically to authenticate 
    the owners of the other keys.
    The currently most used approach is that of Signal, WhatsApp and Telegram.
    They authenticate phone numbers by sending text messages to the phone 
    number.
    Thus they can be sure that the owner of the phone number is also the owner 
    of the private key.
    (Well, the phone operator can of course impersonate the owner of the phone 
    number.)
    Then another user has identified another user by his or her phone number, 
    then the app will ensure the verified key is used.
  \end{solution}

\question[3]
  % tags: E:C:A
  % tags: auth:ac:trustcomp
  What problems do you see for web services to replace passwords with 
  biometrics?
  What would you say is needed for that to work?

  \begin{solution}
    A web service cannot have its own fingerprint reader.
    Thus is cannot know from there whether the fingerprint is fresh.
    This makes the fingerprint equivalent to a password.
    This password being a fingerprint means that is will be the same 
    everywhere, i.e.\ a bad password.

    The web service must be able to ensure freshness.
    This could be done by the web service trusting a particular brand of 
    fingerprint readers.
    Then it can send a challenge to the fingerprint reader which includes the 
    challenge in a digital signature which also signs the fingerprint.

    It could also be implemented such that the user's device is fingerprint 
    protected and then the device simply uses a public-private keypair with the 
    web service.
  \end{solution}


  
\question[3]
% tags: E:C:A
% tags: foundations
% tags: software
Would you classify buffer overruns in software as a violation of 
confidentiality, integrity or availability?
The answer is yes, all three.
Why?

\begin{solution}
  Confidentiality, because you can read out memory you shouldn't.
  Integrity, because you can make the system do things it shouldn't.
  Availability, because you can make the program crash.
\end{solution}



\end{questions}


\printbibliography
\end{document}

