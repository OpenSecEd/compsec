\question[3]
% tags: sidechannels:E:C:A
\begin{parts}
  \part Give an example of a covert channel and
  \begin{solution}
    A server is anonymous (e.g.\ a Tor hidden service), i.e.\ you may access 
    the server but not know its location.
    Part of the server's service is giving the time.
    It has been shown that the variations in the system clock depend on the 
    ambient temperature.
    This means that by studying how the time on the server varies over day and 
    night and over the seasons, we can eventually figure out the ambient 
    temperature.
    From the ambient temperature we can later deduce the geographical location 
    of the server.
  \end{solution}

  \part what we can do about it.
  \begin{solution}
    We can lower the resolution in the time-stamps the server gives, e.g.\ by 
    not giving seconds.
    This lowers the bandwidth of the covert channel, perhaps so that the attack 
    is infeasible.
    We could also sync the servers clock more often, e.g.\ by using the Network 
    Time Protocol.
    However, the only way to prevent it is by not revealing the time of the 
    server's system clock.
  \end{solution}
\end{parts}



\question[3]
  % tags: foundations
  The terms \enquote{data} and \enquote{information} are related but not the 
  same.
  Discuss how they are related and how this affects the term 
  \enquote{security}, as in \enquote{data security} and \enquote{information 
  security}.

  \begin{solution}
  Data is an encoded representation of information.
  Data can be manipulated by formal rules to infer new information.
  E.g.\ the data \enquote{\(x+1 = 0\)} does not directly reveal the value of 
  \(x\), but we can manipulate the data with formal rules (equation solving) 
  and infer the value of \(x\) (\(x = -1\)).

  The same applies for other situations:
  When sending data over an anonymizing network (e.g.\ Tor), we only reveal the 
  size and timing of the packets sent.
  However, these can be statistically correlated to the packets exiting the 
  network, thus the sender and recipient can be inferred.
  \end{solution}
\question[3]
% tags: ac
% tags: E:C:A
\begin{parts}
\part What is discretionary access control?
\part Discuss some of its advantages and disadvantages.
\end{parts}

\begin{solution}
  The owner (creator) of the object may set the access policy for the data 
  object.
  This is what is common in normal file systems and systems like Facebook.

  This puts more responsibility on the user than mandatory access control 
  would: it's the responsibility of the user to analyse and prevent information 
  leaks.
  From a usability perspective, it's of course possible to add a layer on top 
  which does this analysis and presents it to the user.
\end{solution}


\question[3]
  % tags: usability:E:C:A
  Discuss why usability is important for security.

  \begin{solution}
    Då systemen vi är beroende av och som ska upprätthålla vår säkerhet 
    handhas av människor blir psykologin genast viktig.
    Vi behöver psykologin inom säkerhetsområdet för att kunna ta hänsyn till 
    hur människor fungerar när vi konstruerar säkerhetssystem.
    Exempelvis, om vi gör ett system för komplext och användaren tycker att 
    komplexiteten är onödig, då kommer denne användare att aktivt försöka att 
    ta sig runt systemet --- kanske genom att skriva upp långa lösenord 
    istället för att lära sig dem utantill.

    Om vi däremot tar hänsyn till användarnas kognitiva begränsningar, då kan 
    vi konstruera system som både är säkra och enkla att använda.
  \end{solution}
\question
% tags: software:E
Describe the three malware reproduction techniques
\begin{parts}
  \part[1] virus,
  \begin{solution}
    The virus inserts its own code into other programs code.
    When the other programs are run the virus' payload is run too and the 
    infection can spread further.
  \end{solution}

  \part[1] worm,
  \begin{solution}
    The worm spreads by its own means, e.g.\ by utilizing networks (shared file 
    systems, remote executions bugs in network services) or emailing itself 
    using available programs on the computer.
  \end{solution}

  \part[1] trojan horse.
  \begin{solution}
    The trojan horse is a legitimate-looking program which contains unwanted 
    functionality.
    E.g.\ it is a flash-light app, but in the background it uploads the contact 
    list to the app's developer.
  \end{solution}
\end{parts}


\question[3]
  % tags: trustcomp:crypto:E:C:A
  Alice wants to provide confidentiality to a file.
  \begin{parts}
    \part She can accomplish this through mechanisms provided in the operating 
    system.
    Explain how this works and what the limits are.

    \part She can also accomplish this through purely cryptographic mechanisms.
    Explain how this works and what the limits are.
  \end{parts}

  \begin{solution}
    The first way she's securing her file is by using access control mechanisms 
    in the operating system (OS).

    Assuming we have physical access to the computer, then we can just read the 
    raw data from the disk.
    This can be accomplished by either booting our own OS on her computer, or 
    by removing the disk.

    If we don't have physical access we can always try to bypass the access 
    control mechanisms in other ways, e.g.\ by gaining privileges in the system 
    or seeing if the OS has failed to protect reading from the raw disk (i.e.\ 
    not using the file system).

    The main point here is that the operating system must be working correctly 
    for its mechanisms to be effective.
    The \emph{running} operating system will provide confidentiality by not 
    allowing other users' requests to open the file.

    The most obvious way to have system independent security for this file is 
    to encrypt it, i.e.~using cryptographic mechanisms.
    This way no one can read it unless they have access to the key, and this is 
    true no matter if you change the environment.
    (Of course, if the system is untrusted someone can get to the key that way, 
    but that's outside the scope of this question.)
  \end{solution}


  
\question[3]
  % tags: auth:E:C:A
  You work for a start-up company which offers low-cost flat-rate subscriptions 
  for live streaming.
  This will require authentication.
  \begin{parts}
    \part Analyse what data must be authenticated on registration (i.e.\ the 
    set-up of a new subscription).
    Suggest how to do the authentication.

    \begin{solution}
    That the user doing the registration is the user owning the payment data.
    This is usually stipulated by the payment service, e.g.\ Visa or 
    MasterCard: card number as identifier and CVV code as \enquote{password}.

    We must also bootstrap our authentication system in this stage.
    But that only requires that we know that it's the same person paying and 
    setting that up.
    \end{solution}

    \part Analyse what data must be authenticated during streaming.
    Suggest how to do the authentication.

    \begin{solution}
    First, we must to know that the user has a paid subscription.
    Second, we might also need to enforce some age requirements.
    We can use anonymous credentials to reveal as little information as 
    possible.
    \end{solution}
  \end{parts}


\question[3]
  % tags: accountability:E:C:A
  The systems administrator has (due to their role) unlimited access to all 
  computer systems in the organization --- they set up and configure all 
  systems.
  How can you ensure accountability for the systems administrator?

  \begin{solution}
    This can be solved with separation of duties.
    The systems must log to a system that the systems administrator cannot 
    access.
    Thus the systems administrator can be held accountable for what he does.

    Of course, we cannot trust the systems administrator to set this up 
    himself, so we need functional separation.
  \end{solution}


  
\question[3]
  % tags: E:C:A:infotheory:passwd
  Shannon entropy can be used to estimate how well the users choose passwords 
  under a given password composition policy:
  \begin{parts}
    \part How can you estimate the upper bound, i.e.\ the maximum possible 
    entropy?
    And what does the upper bound mean?

    \part Why can't you estimate any (useful) lower bound, i.e.\ the minimum 
    possible entropy?
    What would a lower bound mean?

    \part How can you estimate the actual entropy of when users choose the 
    passwords?
  \end{parts}

  \begin{solution}
    You assume that every part of the password is chosen uniformly randomly.
    This gives the maximum entropy, i.e.~it is an upper bound.
    This means that users cannot do any better than this.
    You have to account for all choices the password composition policy allows.
    Or rather, all choices the policy removes.

    A lower bound would provide a guarantee that users choose passwords 
    sufficiently randomly.
    This is hard because a user can choose a very easy to guess password for 
    any static password policy, which has almost no entropy.
    Similarly, if all users choose the same password, then the entropy would be 
    zero.
    This is why a lower bound is not very useful, it will be zero.

    The average case can be estimated as in~\cite{Komanduri2011opa}.
    You have to \emph{sample a lot of user-generated passwords}, then you can 
    perform a frequency analysis to find the probabilities and compute the 
    entropy.
    The users are the stochastic variable (random output) and you must get 
    a large enough sample to estimate the probability distribution.
  \end{solution}


  
