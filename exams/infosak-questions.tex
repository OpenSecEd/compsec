% $Id$
% Author:  Daniel Bosk <daniel.bosk@miun.se>
\documentclass[a4paper,addpoints]{exam}
\usepackage[utf8]{inputenc}
\usepackage[T1]{fontenc}
\usepackage[swedish,english]{babel}
\usepackage[hyphens]{url}
\usepackage{hyperref}
\usepackage{color}
\usepackage{prettyref,varioref}
\usepackage{subfigure}
\usepackage{amsmath,amssymb}
\usepackage{listings}
\usepackage{authblk}

\usepackage[autopunct]{csquotes}
\MakeBlockQuote{<}{|}{>}
\EnableQuotes{}

\usepackage[natbib,style=alphabetic,maxbibnames=99]{biblatex}
\addbibresource{literature.bib}

%\usepackage[varioref,prettyref,listings]{miunmisc}

\printanswers{}

%\examtype{Final exam}
%\courseid{DV026G}
%\course{Information Security}
\title{Exam Questions in Information Security}
\date{2015-06-01}
\author{%
  Daniel Bosk
  and
  Lennart Franked
}
\affil{%
  Department of Information and Communication Systems\\
  Mid Sweden University, SE-851\,70 Sundsvall
}

\DeclareMathOperator{\hmac}{HMAC}
\DeclareMathOperator{\xor}{\oplus}
\DeclareMathOperator{\concat}{||}

\begin{document}
\maketitle
\thispagestyle{foot}

\section*{Instructions}
\label{sec:Instructions}
Carefully read the questions before you start answering them.
Note the time limit of the exam and plan your answers accordingly.
Only answer the question, do not write about subjects remotely related to the
question.

Write your answers on separate sheets, not on the exam paper.
Only write on one side of the sheets.
Start each question on a new sheet.
Do not forget to \emph{motivate your answers.}

Make sure you write your answers clearly, if I cannot read an answer the answer
will be awarded no points---even if the answer is correct.
The questions are \emph{not} sorted by difficulty.

\begin{description}
  \item[Time] 5 hours.
  \item[Aids] Dictionary, course material and notes.
  \item[Maximum points] \numpoints{}
  \item[Questions] \numquestions{}
\end{description}

\subsection*{Preliminary grades}
The following grading criteria applies:
E \(\geq 50\%\),
D \(\geq 60\%\),
C \(\geq 70\%\), 
B \(\geq 80\%\),
A \(\geq 90\%\).
No question must be awarded zero points.


\clearpage
\section*{Questions}
The questions are given below.
They are not given in any particular order.

\begin{questions}
  \question\label{q:crypto:foundations:E}
  Explain the following terms:
  \begin{parts}
    \part[1] Confidentiality
    \part[1] Integrity
    \part[1] Availability
    \part[1] Accountability
    \part[1] Non-Repudiation
  \end{parts}

  \begin{solution}
    See~\cite{Gollmann2011cs} and~\cite{Anderson2008sea} for definitions.
  \end{solution}


  \question\label{q:trustcomp:crypto:E:C}
  A user wishes to provide confidentiality to a file.
  \begin{parts}
    \part[3] She can accomplish this through mechanisms provided in the 
    operating system.
    Explain how this works and what are the limits.

    \part[3] She can also accomplish this through purely cryptographic 
    mechanisms.
    Explain how this works and what are the limits.
  \end{parts}

  \begin{solution}
    The first way she's securing her file is by using access control mechanisms 
    in the operating system (OS).

    Assuming we have physical access to the computer, then we can just read the 
    raw data from the disk.
    This can be accomplished by either booting our own OS on her computer, or 
    by removing the disk.

    If we don't have physical access we can always try to bypass the access 
    control mechanisms in other ways, e.g.\ by gaining privileges in the system 
    or seeing if the OS has failed to protect reading from the raw disk (i.e.\ 
    not using the file system).

    The main point here is that the operating system must be working correctly 
    for its mechanisms to be effective.
    The \emph{running} operating system will provide confidentiality by not 
    allowing other users' requests to open the file.

    The most obvious way to have system independent security for this file is 
    to encrypt it, i.e.~using cryptographic mechanisms.
    This way no one can read it unless they have access to the key, and this is 
    true no matter if you change the environment.
    (Of course, if the system is untrusted someone can get to the key that way, 
    but that's outside the scope of this question.)
  \end{solution}


  \question\label{q:infotheory:passwd:E:C}
  Explain how information theory can be used to estimate the strength of 
  passwords chosen under a given password composition policy:
  \begin{parts}
    \part[2] How can you estimate the upper bound, i.e.~the maximum possible 
    entropy?

    \part[2] Why can't you estimate any (useful) lower bound, i.e.~the minimum 
    possible entropy?

    \part[2] How can you estimate the average case, i.e.~what is usually the 
    case when users choose the passwords?
  \end{parts}

  \begin{solution}
    You assume that every part of the password is chosen uniformly randomly.
    This gives the maximum entropy, i.e.~it is an upper bound.
    You have to account for all choices the password composition policy allows.
    Or rather, all choices the policy removes.

    This is hard because a user can choose a very easy to guess password, which 
    has almost no entropy.
    Similarly, if all users choose the same password, then the entropy would be 
    zero.

    The average case can be estimated as in~\cite{Komanduri2011opa}.
    You have to \emph{sample a lot of user-generated passwords}, then you can 
    perform a frequency analysis to find the probabilities and compute the 
    entropy.
    The users are the stochastic variable (random output) and you must get 
    a large enough sample to estimate the probability distribution.
  \end{solution}


  \question\label{q:auth:E:C}
  Describe the terms
  \begin{parts}
    \part[2] identification and
    \part[2] authentication.
  \end{parts}
  Make sure to illustrate your explanations by examples.
  You must also give an example of a mechanism for each of the terms.

  \begin{solution}
    In identification you claim an identity.
    This can be done using e.g.~a username, fingerprint or DNA sequence.

    In authentication you prove you are who you claim you are.
    This can be done using e.g.~\emph{who} you are (biometric), \emph{where} 
    you are (location) or what you \emph{do} (biometric), something you 
    \emph{have} (e.g.~BankID), or something you \emph{know} (password).
  \end{solution}


  \question[4]\label{q:accessctrl:E}
  Describe the main differences between Mandatory Access Control (MAC) and 
  Discretionary Access Control (DAC).

  \begin{solution}
    In MAC the access control policy for new material is derived from the 
    subject and objects it is based on.
    One example is Bell-LaPadula (BLP): where information only flows upward, 
    a document produced by someone with clearance \(A\) will also require 
    clearance \(A\) to be read---clearance \(B<A\) will not suffice.
    The system sets the access policy for objects.

    In DAC this is under the control of the owner of an object.
    The owner sets the access policy.
    An example is the UNIX file system.
  \end{solution}


  \question\label{q:accessctrl:E:C}
  You have implemented a Mandatory Access Control (MAC) system in the 
  organization you work for.
  Your goal was to achieve stronger confidentiality for data.
  \begin{parts}
    \part[2] In one sense this is true, explain why.
    \part[2] Some violations of confidentiality simply cannot be protected 
    against, give an example and explain why this is so.
  \end{parts}

  \begin{solution}
    If we add MAC, then we lower the risk of accidentally exposing confidential 
    data since the system would automatically assign the correct classification 
    to the result.
    (This prevents data in high from going to low.)

    A user who is allowed to view data, can always record that data using 
    another device: for data presented visually we can take a photo or 
    videotape it, for sound we can use an audio recorder.
  \end{solution}


  \question\label{q:accessctrl:trustcomp:E:C:A}
  You have implemented a Mandatory Access Control (MAC) system in the 
  organization you work for.
  \begin{parts}
    \part[4] Explain what properties you might have wanted which made you do 
    this.

    \part[2] Some employees use laptops.
    If you only rely upon the operating system to enforce the policy through 
    MAC, then those employees with laptops might be able to break your policy.
    Give an example of how a user can break your security policy in this 
    situation.
    Make clear what part of your policy is violated and why it can be done.
  \end{parts}

  \begin{solution}
    The most obvious property is confidentiality.
    If we add MAC, then we lower the risk of accidentally exposing confidential 
    data since the system would automatically assign the correct classification 
    to the result.
    (This prevents data in high from going to low.)

    Another property which is plausible is integrity.
    This way the system ensures that we don't trust data more than its source.
    E.g.\ if we download a file from the network, then we shouldn't trust it 
    more than we trust the network.
    This means that the system can e.g.\ disallow executing such a file.
    (We only allow execution of trusted data.)

    A user with a laptop has access to the hardware.
    This means that the user can bypass the operating system.
    By doing that the user can read and write arbitrary data from and to the 
    hard-disk.
  \end{solution}


  \question\label{q:accountability:E:C}
  Separation of duties is a core concept for security.
  \begin{parts}
    \part[2] Describe the two types of separation of duties.
    \part[1] What is the main reason for separation of duties?
  \end{parts}

  \begin{solution}
    There are two types of separation of duties:
    dual control and functional separation.
    Dual control means that two or more subjects must act together (at the same 
    time) to authorize a transaction.
    Functional separation means that several functions are needed to authorize 
    a transaction---e.g.~create a transaction and verify it---and one subject 
    is not allowed to do both functions.

    The reason for separation of duties to make it impossible for one malicious 
    subject to compromise a system.
    With separation of duties the malicious subject must persuade one or more 
    other subjects to collude.
  \end{solution}


  \question\label{q:passwd:auth:E}
  Explain the following terms:
  \begin{parts}
    \part[1] Brute force
    \part[1] Dictionary attack
    \part[1] Hash table
    \part[1] Social engineering
    \part[1] Two-factor authentication
    \part[1] Phishing
  \end{parts}


%  \%question\label{q:msb}
%  Explain the following terms:
%  \begin{parts}
%    \part[1] GAP-analys
%    \part[1] LIS
%    \part[1] säkerhetspolicy
%    \part[1] nulägesanalys
%    \part[1] ISO 27000
%  \end{parts}


  \question\label{q:passwd:infotheory:E:C:A}
  You are asked to estimate some password policies.
  The policies are the following:
  \begin{description}
    \item[comprehensive8]
      At least 8 characters consisting of upper and lower case, numbers and 
      special characters (assume the ones common with the numbers on 
      a keyboard).
    \item[randswedict3]
      Randomly choose three words from the Dictionary of the Swedish Language 
      (SAOL).
      This dictionary contains approximately 125\,000 words.
  \end{description}
  You should answer the following:
  \begin{parts}
    \part[4] Estimate the entropy for the passsword policies.
    (You may rely on the results in certain published research papers discussed 
    in the course for certain estimates.)
    \part[2] Decide how suitable they are for use in the home environment.
    \part[2] Decide how suitable they are for use in a web application.
  \end{parts}
  Note that you will not get any points without a motivation.


  \question\label{q:passwd:infotheory:E:C:A}
  You are asked to estimate some password policies.
  The policies are the following:
  \begin{description}
    \item[comprehensive8]
      At least 8 characters consisting of upper and lower case, numbers and 
      special characters (assume the ones common with the numbers on 
      a keyboard).
    \item[randswedict3]
      Randomly choose three words from the Dictionary of the Swedish Language 
      (SAOL).
      This dictionary contains approximately 125\,000 words.
  \end{description}
  You should answer the following:
  \begin{parts}
    \part[4] Estimate the entropy for the passsword policies.
    (You may rely on the results in certain published research papers discussed 
    in the course for certain estimates.)
    \part[2] Decide how suitable they are for use in a large organization.
    \part[2] Decide how suitable they are for use in a web application.
  \end{parts}
  Note that you will not get any points without a motivation.


  \question\label{q:passwd:infotheory:E:C:A}
  You are asked to estimate some password policies.
  The policies are the following:
  \begin{description}
    \item[basic12]
      At least 12 characters consisting of upper and lower case, and numbers.
    \item[randswedict4]
      Randomly choose four words from the Dictionary of the Swedish Language 
      (SAOL).
      This dictionary contains approximately 125\,000 words.
  \end{description}
  You should answer the following:
  \begin{parts}
    \part[4] Estimate the entropy for the passsword policies.
    (You may rely on the results in certain published research papers discussed 
    in the course for certain estimates.)
    \part[2] Decide how suitable they are for use in the home environment.
    \part[2] Decide how suitable they are for use in a web application.
  \end{parts}
  Note that you will not get any points without a motivation.


  \question\label{q:passwd:infotheory:E:C:A}
  You are asked to estimate some password policies.
  The policies are the following:
  \begin{description}
    \item[basic12]
      At least 12 characters consisting of upper and lower case, and numbers.
    \item[randswedict4]
      Randomly choose four words from the Dictionary of the Swedish Language 
      (SAOL).
      This dictionary contains approximately 125\,000 words.
  \end{description}
  You should answer the following:
  \begin{parts}
    \part[4] Estimate the entropy for the passsword policies.
    (You may rely on the results in certain published research papers discussed 
    in the course for certain estimates.)
    \part[2] Decide how suitable they are for use in a large organization.
    \part[2] Decide how suitable they are for use in a web application.
  \end{parts}
  Note that you will not get any points without a motivation.


  \question[5]\label{q:msb:E:C}
  When you start in the company you realize that there is no Information 
  Security Management System (ISMS, Swe.~`ledningssystem för 
  informationssäkerhet').
  You go into your boss's office: persuade him and the management that you need 
  an ISMS\@.


  \question[5]\label{q:msb:E:C}
  Your boss is finally convinced that the company needs an Information Security 
  Management System (ISMS, Swe.~`ledningssystem för informationssäkerhet').
  He comes to ask you how an ISMS is best implemented, explain how that is 
  done.


%  \%question\label{q:pwdpolicy}
%  Universitetets lösenordspolicy kräver minst åtta tecken.
%  Dessa tecken ska vara minst tre gemener, tre versaler och två siffror -- 
%  dessutom måste dessa finnas bland de första åtta tecknen i lösenordet.
%  Detta ger \( 26^3 26^3 10^2 = 26^6 10^2 \approx 2^{35} \) antal möjliga 
%  lösenord.
%  Lösenordet måste dessutom bytas var tredje månad, vilket i sin tur ger risken 
%  för lösenordssystem där användaren baserar det nya lösenordet på det gamla.
%
%  Den något enklare lösenordspolicyn minst åtta tecken med gemener, versaler 
%  och siffror -- utan krav på något antal inom de olika kategorierna -- ger \( 
%  (26 + 26 + 10)^8 = 62^8 \approx 2^{48} \) antal lösenord.
%  Denna policy har inga krav på giltighetstid hos ett lösenord.
%
%  Resultatet av detta är en reduktion av komplexiteten från \( 62^8 \) ned till 
%  \( 26^6 10^2 \).
%  Detta utgör en relativ minskning av komplexiteten med \( 1-\frac{26^6 
%  10^2}{62^8} = 0.99986 \), alltså 99.99 procent.
%
%  Oavsett vilken av ovan givna policyer som används får användarna (sannolikt) 
%  svaga lösenord.
%
%  \begin{parts}
%    \part[4] Förklara hur dessa lösenordspolicyer kan angripas.
%
%    \begin{solution}
%      De kan naturligtvis angripas genom en ordlisteattack.
%      Eftersom att den första policyn har den givna utformningen behöver 
%      ordlistan inte innehålla några gissningsförslag som innehåller 
%      specialtecken.
%      (Den bör inte innehålla specialtecken inom de första åtta tecknen för att 
%      öka hastigheten för gissningarna, det är ju onödigt att gissa på något 
%      som ändå inte tillåts enligt policyn.)
%
%      Ett annat angreppssätt är råstyrkemetoden.
%      Här kan tas hänsyn till att lösenordet är minst åtta tecken, och 
%      sannolikt inte är särskilt mycket längre; att antalet versaler, gemener 
%      och siffror är av givna antal.
%      Om vi har ett system som klarar av \( 10^6\approx 2^{20} \) gissningar 
%      per sekund och lösenordsrymden är cirka \( 2^{35} \), då har vi gått 
%      igenom hela lösenordsrymden på \( 2^{15} \) sekunder, tillika nio timmar.
%      Antar vi det mycket långsammare systemet med \( 10^3\approx 2^{10} \), 
%      tar det istället 388 dagar.
%
%      Båda angreppsmetoderna kan förfinas ytterligare genom att anta att 
%      användarna använder någon form av system för utformandet av sina 
%      lösenord, detta antagande baserar sig på det påtvingade lösenordsbytet 
%      var tredje månad.
%      (Detta gäller naturligtvis enbart den första policyn, då den andra inte 
%      har några krav på lösenordsbyten.)
%      Ordlistemetoden kan utökas med tillägg av diverse prefix eller suffix på 
%      varje ord i ordlistan.
%      På liknande sätt kan råstyrkeattacken anpassas för att testa lösenord 
%      i en bättre ordning och förhoppningsvis avsluta innan sökrymden är 
%      uttömd.
%    \end{solution}
%
%    \part[4] Ge ett förslag på en riktigt bra lösenordspolicy.
%    Förslag utan motivering ger noll poäng.
%
%    \begin{solution}
%      De problem som uppstår med ovan givna lösenordspolicyer är dels att 
%      lösenordsrymden inskränks, speciellt av den första policyn.
%      Det andra problemet är lösenordsbytet, vilket kan få användarna att 
%      tillämpa system för lösenordsbytena; exempelvis att öka på en siffra 
%      i slutet av lösenordet.
%      Anledningen till detta är minnesbarhet.
%      Användarna vill ha ett lösenord som är enkelt att komma ihåg, har de lärt 
%      sig ett lösenord och sedan gör en liten modifiering då blir det enklare 
%      att komma ihåg än att lära sig ett helt nytt lösenord.
%
%      I en undersökning av \citet{Komanduri2011opa} visar det sig att den 
%      lösenordspolicy som ger högst entropi är den som enbart kräver minst 16 
%      tecken.
%      Entropin används här för att beteckna den faktiska lösenordsrymden, många 
%      lösenordskombinationer kommer aldrig att användas och entropin är ett 
%      mått på just hur många som faktiskt kommer att användas.
%      Denna lösenordspolicy ger också en av de högsta minnesbarheterna 
%      i undersökningen.
%
%      Om vi antar att användare kommer att välja en väldigt enkelt lösenord 
%      enligt denna policy, exempelvis bara 16 gemener, då kommer sökrymden för 
%      en råstyrkeattack att bli \( 26^{16}\approx 2^{75} \).
%      Med vår snabbaste angripar ovan kommer det alltså att ta \( 10^{15} \) år 
%      att gå igenom hela sökrymden.
%      Notera att detta är ett minimum då medellängden för lösenorden enligt 
%      denna policy var längre än 16 tecken \cite{Komanduri2011opa}.
%
%      Låt oss anta att medellängden för ord i SAOL \cite{SAOL} är fyra tecken.
%      För att kunna ha en svensk mening som lösenord måste vi då ha en mening 
%      beståendes av minst fyra svenska ord.
%      Om vi då använder SAOL för att angripa dessa lösenord, då har vi att 
%      sökrymden är \( 125000^4\approx 2^{67} \) och det skulle fortfarande ta 
%      den snabbaste angriparen ovan \( 10^{12} \) år att gå igenom alla 
%      gissningarna.
%
%      Utifrån detta kan vi konstatera att vi egentligen inte skulle behöva 
%      någon livslängd för lösenord.
%      Problemen som kan uppstå är om någon iaktar när lösenordet skrivs in 
%      eller dylika problem.
%      Därför kan det vara bra att införa en livslängd för lösenord på ett till 
%      fem år, då blir det mindre frustration för användarna men ändå viss 
%      förnyelse av lösenorden.
%    \end{solution}
%  \end{parts}


  \question\label{q:foundations:E}
  Define the following terms:
  \begin{parts}
    \part[1] Trusted
    \part[1] Trustworthy
    \part[1] Secrecy
    \part[1] Confidentiality
    \part[1] Privacy
    \part[1] Integrity
    \part[1] Authenticity
  \end{parts}

  \begin{solution}
    \citet[ss.\ 13--14]{Anderson2008sea} definierar begreppen enligt följande:
    \begin{description}
      \item[Pålitlighet] Ett system eller principal som innehar pålitlighet 
        (\foreignlanguage{english}{is trusted}) är ett system eller principal 
        som kan bryta din säkerhetspolicy.

      \item[Pålitlig] Ett system eller principal som är pålitlig 
        (\foreignlanguage{english}{is trustworthy}) är ett system eller 
        principal som inte kommer att misslyckas.
        (Den kommer alltså inte att bryta din säkerhetspolicy.)

        Ett exempel för att illustrera skillnaden ges av följande citat: 
        \blockcquote[s.\ 13]{Anderson2008sea}{%
          if an NSA employee is observed in a toilet stall at Baltimore 
          Washington airport selling key material to a Chinese diplomat, then 
          (assuming his operation was not authorized) we can describe him as 
          \enquote{trusted but not trustworthy}%
        }.

      \item[Sekretess] Sekretess är en teknisk term för effekten av en mekanism 
        som begränsar antalet principals som kan ta del av information.

      \item[Konfidentialitet] Konfidentialitet syftar till att tillhandahålla 
        sekretess för andra principals hemliga information.

      \item[Personlig integritet] Detta är förmågan eller rätten att kunna 
        skydda sin personliga information.
        Det gäller alltså bara individer, exempelvis företag har ingen 
        personlig integritet.

      \item[Integritet] Detta är en teknisk term för egenskapen att data 
        förblir oförändrat, eller, om förändring sker ska den inte förbli 
        obemärkt.

      \item[Autenticitet] Detta begrepp innefattar integritet och fräshhet.
        Om kommunikation spelas in och sedan spelas upp vid ett annat 
        tillfälle, då kommer integriteten att ha bevarats men inte fräshheten 
        --- alltså är en återuppspelning inte autentisk.
    \end{description}
    Dessa definitioner stämmer även överens med RFC 4949~\cite{rfc4949}.
  \end{solution}


  \question\label{q:usability:E:C:A}
  Human psychology is important in security.
  It is used in both security usability and social engineering.
  \begin{parts}
    \part[2] Give an overview of why psychology is important in 
    security.

    \begin{solution}
      Då systemen vi är beroende av och som ska upprätthålla vår säkerhet 
      handhas av människor blir psykologin genast viktig.
      Vi behöver psykologin inom säkerhetsområdet för att kunna ta hänsyn till 
      hur människor fungerar när vi konstruerar säkerhetssystem.
      Exempelvis, om vi gör ett system för komplext och användaren tycker att 
      komplexiteten är onödig, då kommer denne användare att aktivt försöka att 
      ta sig runt systemet --- kanske genom att skriva upp långa lösenord 
      istället för att lära sig dem utantill.

      Om vi däremot tar hänsyn till användarnas kognitiva begränsningar, då kan 
      vi konstruera system som både är säkra och enkla att använda.
    \end{solution}

    \part[4] Give an example of an attack which exploits weaknesses in human 
    psychology.
    Also explain why it works.

    \begin{solution}
      En psykologibaserad attack utnyttjar svagheter hos användarna för att ta 
      sig runt ett säkerhetssystem, det är alltså inte säkerhetssystemen som 
      angrips.

      Ett exempel på en sådan attack kan vara att en användare får ett e-brev 
      som till synes är från banken och som innehåller en länk till en 
      inloggningssida, kallat nätfiske.
      Brevet kan be användaren att uppdatera någonting hos banken via internet.
      Ett förfarande beskrivs och sedan läggs till \enquote{eller klicka på 
        länken}.
      Med en förfarande som låter som att det kan ta fem till tio klick kommer 
      användaren sannolikt att välja enklicksalternativet.
      Notera att förfarandet måste vara korrekt för banken medan länken är till 
      en phishingsida.
      Utformandet kan leda till vad litteraturen~\cite[s. 23]{Anderson2008sea} 
      kallar \emph{\foreignlanguage{english}{capture errors}}, att användaren 
      använder ett invant beteende: i detta fall att användaren klickar på 
      direktlänkar.

      Därutöver försöker nätfiskaren att få användaren att tillämpa fel regler 
      i situationen.
      Exempelvis, användaren kanske (omedvetet) lägger större vikt vid att ett 
      hänglås syns i webbläsaren för säker anslutning än att bankens namn är 
      rätt stavat i URL:en.
      Även att bankens namn finns med någonstans i URL:en kan vara en 
      tillräckligt stark regel för att användaren ska undvika att detektera den 
      felaktiga fiske-URL:en.
    \end{solution}
  \end{parts}


  \question\label{q:lvlltrl:E:C:A}
  Multi-level security and multi-lateral security are two related terms.
  \begin{parts}
    \part[3] Explain what multi-level security is and what we want to 
    accomplish with it.

    \part[3] Explain what multi-lateral security is and what we want to 
    accomplish with it.

    \part[3] Give some of the advantages of combining them and explain why this 
    is so.
  \end{parts}

  \begin{solution}
    Begreppet \emph{flernivåsäkerhet} innebär att information eller åtkomst 
    därutav värderas olika i en hierarkisk modell~\cite[kap.\ 
    8]{Anderson2008sea}.
    Två välkända exempel är Bell--LaPadula-modellen (BLP) och Bibamodellen.
    BLP fokuserar på konfidentialitet och information värderas som mer eller 
    mindre konfidentiell.
    Den går ut på att en principal som har en given åtkomstbehörighet kan läsa 
    allt från sin egen nivå och alla nivåer nedan.
    Däremot kan samma användare enbart skriva på sin egen nivå eller på övre 
    nivåer.
    Detta är för att hindra informationsflöde från högre nivåer till lägre.

    Bibamodellen fungerar tvärt om då den fokuserar på att bibehålla 
    integriteten hos information istället för konfidentialiteten.

    \emph{Multilateral säkerhet}, å andra sidan, är inte hierarkisk.
    Den delar upp information horisontellt i olika avdelningar och det ska vara 
    omöjligt att information flödar från en avdelning till en annan.
    Välkända modeller inom multilateral säkerhet är Kinesiska muren och 
    Separation av uppgifter (\foreignlanguage{english}{Separation of duty}).

    Kinesiska muren innebär att det mellan varje avdelning skapas en 
    \enquote{mur} för att information inte ska kunna flöda emellan.
    Den innebär att om en anställd arbetar med en kund får den anställde inte 
    arbeta med kunder som är konkurrerande.

    Separation av uppgifter innebär exempelvis att det inte får vara samma 
    personer som driftar ett system som de som utvecklar det.
    Anledningen till detta är för att utvecklarna inte ska medvetet kunna bygga 
    in svagheter i systemet och sedan utnyttja dessa.
    Ett annat exempel är att en systemadministratör inte ska ha 
    skrivrättigheter till loggfilerna för ett system denne själv administrerar, 
    då denne i sådant fall skulle kunna göra vadhelst och sedan ta bort det 
    från loggarna.

    Fördelarna med att kombinera flernivå- och multilateral säkerhet är att ett 
    sådant system ger möjlighet till mer finskaliga säkerhetspolicyer.
    Exempelvis ska en läkare ha tillgång till patientjournaler, men enbart för 
    de patienter som denne ansvarar för.
    På samma sätt ska receptionisten kunna komma åt besökstider för alla 
    patienter (vid en given vårdinrättning), men inte journalerna.
  \end{solution}


  \question[5]\label{q:biometrics:E:C:A}
  Describe what problems can arise when using biometrical systems.
  Explain why they still can be used if employed properly.

  \begin{solution}
    \citet{Anderson2008sea} anger några problem med biometriska system, 
    sammanfattningsvis är de
    \begin{itemize}
      \item att systemen fungerar dåligt eller inte alls för personer med 
        kroppskador, då systemen är anpassade för mycket begränsad naturlig 
        variation;
      \item att många system går att lura med hjälp av inspelningsattacker, 
        exempelvis att ett gammalt fingeravtryck visas upp för 
        fingeravtrycksläsaren eller att en inspelning spelas upp för ett 
        röstigenkänningsystem;
      \item att det är svårt för en mänsklig operatör att verifiera eller 
        falsifiera biometriska data;
      \item att precisionen i det biometriska systemet är för dålig för att 
        kunna skilja en användare från en annan.
    \end{itemize}

    En anledning till att de ändå kan användas är för att de kan ha en 
    avskräckande effekt, alltså en ren psykologisk effekt hos angripare (och 
    hos användare generellt).
    De kan även användas i kombination med en annan säkerhetsmekanism, 
    exempelvis ett lösenord, för att åstadkomma tvåfaktorautentisering.
  \end{solution}


  \question\label{q:trustcomp:C:A}
  The company you work for want to implement extra features as in-app purchases 
  for the company's main product.
  You are currently in a meeting about that particular topic, the chairperson 
  of the meeting points at you and asks:
  \enquote{%
    How would you design this system?
    The customers must pay for the features, for every installation, we cannot 
    allow them to just buy once and copy later.
    Give us an overview.
  }
  \begin{parts}
    \part[2] Outline the main points in your strategy.
    \part[2] There are some things you simply cannot protect against.
    Explain the limits of systems such as these, so that everyone present 
    understands the limits.
  \end{parts}

  \begin{solution}
    If possible, put the features on a server and execute them remotely.
    This requires proper authentication.
    Ensure this authentication cannot be broken, so that two devices can use 
    the same credentials.
    Try to use a platform where a user must \enquote{root} their phones to 
    violate this.

    The limitations of these systems are that they try to protect code running 
    in a hostile environment.
    If the user has total control of the hardware, then the user can simply 
    copy the software to another device.
    The user can even make arbitrary changes to the running code.
    This means that any keys can be copied from one device to another.
    Even if the app is programmed to transmit the device ID, the app can be 
    reprogrammed to transmit a hard-coded static device ID\@.
  \end{solution}

\end{questions}


\printbibliography{}
\end{document}
