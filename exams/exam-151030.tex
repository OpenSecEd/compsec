% $Id$
% Author:  Daniel Bosk <daniel.bosk@miun.se>
\documentclass[svv,addpoints]{miunexam}
\usepackage[utf8]{inputenc}
\usepackage[T1]{fontenc}
\usepackage[swedish,english]{babel}
\usepackage[hyphens]{url}
\usepackage{hyperref}
\usepackage{color}
\usepackage{prettyref,varioref}
\usepackage{subfigure}
\usepackage{amsmath,amssymb}
\usepackage{listings}
\usepackage{authblk}

\usepackage{csquotes}
\MakeBlockQuote{<}{|}{>}
\EnableQuotes

\usepackage[natbib,style=alphabetic,maxbibnames=99,backend=bibtexu]{biblatex}
\addbibresource{literature.bib}

\usepackage[varioref,prettyref,listings]{miunmisc}

%\printanswers

\examtype{Final exam}
\courseid{DT145G}
\course{Computer Security}
\date{2015-10-30}
\author{%
  Daniel Bosk
}
\affil{%
  Department of Information and Communication Systems,\\
  Mid Sweden University, SE-851\,70 Sundsvall\\
  Email: \href{mailto:daniel.bosk@miun.se}{daniel.bosk@miun.se}\\
  Phone: 010-142\,8709
}

\DeclareMathOperator{\hmac}{HMAC}
\DeclareMathOperator{\xor}{\oplus}
\DeclareMathOperator{\concat}{||}

\begin{document}
\maketitle
\thispagestyle{foot}

\section*{Instructions}
\label{sec:Instructions}
Carefully read the questions before you start answering them.
Note the time limit of the exam and plan your answers accordingly.
Only answer the question, do not write about subjects remotely related to the
question.

Write your answers on separate sheets, not on the exam paper.
Only write on one side of the sheets.
Start each question on a new sheet.
Do not forget to \emph{motivate your answers.}

Make sure you write your answers clearly, if I cannot read an answer the answer
will be awarded no points---even if the answer is correct.
The questions are \emph{not} sorted by difficulty.

\begin{description}
  \item[Time] 5 hours.
  \item[Aids] Dictionary.
  \item[Maximum points] \numpoints
  \item[Questions] \numquestions
\end{description}

\subsection*{Preliminary grades}
The following grading criteria applies:
E \(\geq 50\%\),
D \(\geq 60\%\),
C \(\geq 70\%\), 
B \(\geq 80\%\),
A \(\geq 90\%\).


\clearpage
\section*{Questions}
The questions are given below.
They are not given in any particular order.

\begin{questions}
\question\label{q:crypto:foundations:E}
  Explain the following terms:
  \begin{parts}
    \part[1] Confidentiality
    \part[1] Integrity
    \part[1] Availability
    \part[1] Accountability
    \part[1] Non-Repudiation
  \end{parts}

  \begin{solution}
    See~\cite{Gollmann2011cs} and~\cite{Anderson2008sea} for definitions.
  \end{solution}


  

\question\label{q:drm:trustcomp:A}
  The apps in the Google Play store or Apple's AppStore are provided with some 
  DRM, e.g.~you cannot copy a paid-for app from one phone to another etc.
  There are also other things you are not allowed to do with these smartphones, 
  e.g.~installing whatever software you like, modifying the software in any way 
  you like, etc.
  (However, many know of the terms <jail-breaking> or <rooting> their phones, 
  which bypasses these mechanisms on the phones.)

  \begin{parts}
    \part[3] Explain how this type of protection system works, including the 
    fundamental assumptions needed for it to fulfil its purpose.

    \part[3] Explain how the above system can be broken and whether it could be 
    fully solved or not.
  \end{parts}
  
  \begin{solution}
    The fundamental assumption is based on the fact that the operating system 
    remains in control and that it cannot be modified.
    This is accomplished by the operating system being the only provided 
    interface with which the user can interact with the device.

    Essentially the user is allowed to use the operating system as an 
    unprivileged user, whereas the system keeps superuser (or root) privileges 
    for itself.
    Hence, as soon as the user gains these privileges, the system breaks down.
    This can be accomplished in many ways, e.g.~exploiting bugs in privileged 
    applications or the OS itself---and is thus called <rooting the device>.
  \end{solution}


  
\question\label{q:passwd:auth:E}
  Explain the following terms:
  \begin{parts}
    \part[1] Brute force
    \part[1] Dictionary attack
    \part[1] Hash table
    \part[1] Social engineering
    \part[1] Two-factor authentication
    \part[1] Phishing
  \end{parts}


 

\question\label{q:usability:E:C:A}
  Human psychology is important in security.
  It is used in both security usability and social engineering.
  \begin{parts}
    \part[2] Give an overview of why psychology is important in 
    security.

    \begin{solution}
      Då systemen vi är beroende av och som ska upprätthålla vår säkerhet 
      handhas av människor blir psykologin genast viktig.
      Vi behöver psykologin inom säkerhetsområdet för att kunna ta hänsyn till 
      hur människor fungerar när vi konstruerar säkerhetssystem.
      Exempelvis, om vi gör ett system för komplext och användaren tycker att 
      komplexiteten är onödig, då kommer denne användare att aktivt försöka att 
      ta sig runt systemet --- kanske genom att skriva upp långa lösenord 
      istället för att lära sig dem utantill.

      Om vi däremot tar hänsyn till användarnas kognitiva begränsningar, då kan 
      vi konstruera system som både är säkra och enkla att använda.
    \end{solution}

    \part[4] Give an example of an attack which exploits weaknesses in human 
    psychology.
    Also explain why it works.

    \begin{solution}
      En psykologibaserad attack utnyttjar svagheter hos användarna för att ta 
      sig runt ett säkerhetssystem, det är alltså inte säkerhetssystemen som 
      angrips.

      Ett exempel på en sådan attack kan vara att en användare får ett e-brev 
      som till synes är från banken och som innehåller en länk till en 
      inloggningssida, kallat nätfiske.
      Brevet kan be användaren att uppdatera någonting hos banken via internet.
      Ett förfarande beskrivs och sedan läggs till \enquote{eller klicka på 
        länken}.
      Med en förfarande som låter som att det kan ta fem till tio klick kommer 
      användaren sannolikt att välja enklicksalternativet.
      Notera att förfarandet måste vara korrekt för banken medan länken är till 
      en phishingsida.
      Utformandet kan leda till vad litteraturen~\cite[s. 23]{Anderson2008sea} 
      kallar \emph{\foreignlanguage{english}{capture errors}}, att användaren 
      använder ett invant beteende: i detta fall att användaren klickar på 
      direktlänkar.

      Därutöver försöker nätfiskaren att få användaren att tillämpa fel regler 
      i situationen.
      Exempelvis, användaren kanske (omedvetet) lägger större vikt vid att ett 
      hänglås syns i webbläsaren för säker anslutning än att bankens namn är 
      rätt stavat i URL:en.
      Även att bankens namn finns med någonstans i URL:en kan vara en 
      tillräckligt stark regel för att användaren ska undvika att detektera den 
      felaktiga fiske-URL:en.
    \end{solution}
  \end{parts}


  
\question\label{q:accountability:E:C}
  Separation of duties is a core concept for security.
  \begin{parts}
    \part[2] Describe the two types of separation of duties.
    \part[1] What is the main reason for separation of duties?
  \end{parts}

  \begin{solution}
    There are two types of separation of duties:
    dual control and functional separation.
    Dual control means that two or more subjects must act together (at the same 
    time) to authorize a transaction.
    Functional separation means that several functions are needed to authorize 
    a transaction---e.g.~create a transaction and verify it---and one subject 
    is not allowed to do both functions.

    The reason for separation of duties to make it impossible for one malicious 
    subject to compromise a system.
    With separation of duties the malicious subject must persuade one or more 
    other subjects to collude.
  \end{solution}


  
\question\label{q:accessctrl:E:C}
  You have implemented a Mandatory Access Control (MAC) system in the 
  organization you work for.
  Your goal was to achieve stronger confidentiality for data.
  \begin{parts}
    \part[2] In one sense this is true, explain why.
    \part[2] Some violations of confidentiality simply cannot be protected 
    against, give an example and explain why this is so.
  \end{parts}

  \begin{solution}
    If we add MAC, then we lower the risk of accidentally exposing confidential 
    data since the system would automatically assign the correct classification 
    to the result.
    (This prevents data in high from going to low.)

    A user who is allowed to view data, can always record that data using 
    another device: for data presented visually we can take a photo or 
    videotape it, for sound we can use an audio recorder.
  \end{solution}


  
\question\label{q:passwd:infotheory:E:C:A}
  You are asked to estimate some password policies.
  The policies are the following:
  \begin{description}
    \item[basic12]
      At least 12 characters consisting of upper and lower case, and numbers.
    \item[randswedict4]
      Randomly choose four words from the Dictionary of the Swedish Language 
      (SAOL).
      This dictionary contains approximately 125\,000 words.
  \end{description}
  You should answer the following:
  \begin{parts}
    \part[4] Estimate the entropy for the passsword policies.
    (You may rely on the results in certain published research papers discussed 
    in the course for certain estimates.)
    \part[2] Decide how suitable they are for use in the home environment.
    \part[2] Decide how suitable they are for use in a web application.
  \end{parts}
  Note that you will not get any points without a motivation.


  
\question\label{q:software:stacksmash:C}
  Look at the C code in \prettyref{lst:overrun}.
  \begin{parts}
    \part[3] Identify all vulnerabilities in that code and motivate by stating 
    how they can be exploited.

    \part[3] Suggest improvements to remedy these vulnerabilities, you must 
    motivate why they work.
  \end{parts}
  \begin{solution}
    There is a possibility for stack smashing in \code{get_some_input} and 
    \code{main}.
    Check the boundaries of buffers, never let scanf(3) read more data than the 
    buffer can hold.

    There is an integer overflow in \code{make_full_name}.
    A size can never be negative, hence don't use signed integers.
    Use the \code{size_t} data type, which is defined as unsigned.
    This way we can only overflow using twice the data.
    Another way is to check if we expect an overflow to occur, e.g.~if both 
    sizes are larger than half of the maximum value.
  \end{solution}

  \begin{src}[float,caption={Some vulnerable C code.},label={lst:overrun}]
#include <stdio.h>

int
get_some_input( void )
{
  char buffer[128];

  printf( "Please enter the key: " );
  scanf( "%s", buffer );

  /* process input */

  return 0;
}

void
make_full_name( char *dst, int dstlen,
                const char *src, int srclen,
                int maxsize )
{
  if ( dstlen + srclen + 1 >= maxsize )
    return -1;

  strncat( dst, " ", 1 );
  return strncat( dst, src, srclen );
}

int
main( int argc, char **argv )
{
  char first[256];
  char last[256];

  printf( "Please enter your first name: " );
  scanf( "%s", first );
  printf( "Please enter you last name: " );
  scanf( "%s", last );

  make_full_name( first, strlen( first ),
                  last, strlen( last ), 256 );

  if ( get_some_input() < 0 )
    return -1;

  return 0;
}
  \end{src}


\end{questions}


\printbibliography
\end{document}

