\question[3]
% tags: auth:E:C:A
  Describe the terms identification and authentication and how they relate to 
  each other.
  Make sure to illustrate your explanations with examples.

  \begin{solution}
    In identification you claim an identity.
    This can be done using e.g.\ a username, fingerprint or DNA sequence.
    Anyone can do this.

    In authentication you prove you are who you claim you are.
    This can be done using e.g.\ \emph{who} you are (biometric), \emph{where} 
    you are (location) or what you \emph{do} (biometric), something you 
    \emph{have} (e.g.~BankID), or something you \emph{know} (password).
    One important part of authentication is freshness.
  \end{solution}


  
\question[3]
% tags: foundations:E:C:A
Discuss the relation between the following pairs of terms:
  \begin{parts}
    \part Trusted and trustworthy
    \part Confidentiality and privacy
    \part Integrity and authenticity
  \end{parts}

  \begin{solution}
    \citet[ss.\ 13--14]{Anderson2008sea} definierar begreppen enligt följande:
    \begin{description}
      \item[Pålitlighet] Ett system eller principal som innehar pålitlighet 
        (\foreignlanguage{english}{is trusted}) är ett system eller principal 
        som kan bryta din säkerhetspolicy.

      \item[Pålitlig] Ett system eller principal som är pålitlig 
        (\foreignlanguage{english}{is trustworthy}) är ett system eller 
        principal som inte kommer att misslyckas.
        (Den kommer alltså inte att bryta din säkerhetspolicy.)

        Ett exempel för att illustrera skillnaden ges av följande citat: 
        \blockcquote[s.\ 13]{Anderson2008sea}{%
          if an NSA employee is observed in a toilet stall at Baltimore 
          Washington airport selling key material to a Chinese diplomat, then 
          (assuming his operation was not authorized) we can describe him as 
          \enquote{trusted but not trustworthy}%
        }.

      \item[Sekretess] Sekretess är en teknisk term för effekten av en mekanism 
        som begränsar antalet principals som kan ta del av information.

      \item[Konfidentialitet] Konfidentialitet syftar till att tillhandahålla 
        sekretess för andra principals hemliga information.

      \item[Personlig integritet] Detta är förmågan eller rätten att kunna 
        skydda sin personliga information.
        Det gäller alltså bara individer, exempelvis företag har ingen 
        personlig integritet.

      \item[Integritet] Detta är en teknisk term för egenskapen att data 
        förblir oförändrat, eller, om förändring sker ska den inte förbli 
        obemärkt.

      \item[Autenticitet] Detta begrepp innefattar integritet och fräshhet.
        Om kommunikation spelas in och sedan spelas upp vid ett annat 
        tillfälle, då kommer integriteten att ha bevarats men inte fräshheten 
        --- alltså är en återuppspelning inte autentisk.
    \end{description}
    Dessa definitioner stämmer även överens med RFC 4949~\cite{rfc4949}.
  \end{solution}


  
\question[3]\label{q:sidechannels}
% tags: sidechannels:E:C:A
Describe an attack scenario where a side-channel is of central interest.

\begin{solution}
  The adversary is interested in learning classified information.
  They set up a device which records electromagnetic emissions to reconstruct 
  the image on a screen, thus when a target works with the classified data on 
  the computer the adversary sees the same image.
  This is a passive attack.
\end{solution}

\question[3]
% tags: trustcomp:crypto
% tags: E:C:A
  Explain why it is not possible to \emph{securely} embedd a cryptographic key 
  in a program (e.g. a smartphone app), i.e.\ why the adversary can steal the 
  key.
  Also describe the requirements for a solution and possibly some example.

  \begin{solution}
    Since the software might run in a hostile environment, the adversary can 
    extract the key.

    The only way to prevent an adversary from extracting the key is to store it 
    in special-purpose hardware.
    If you can ensure the integrity of the operating system, then sandboxing 
    the applications will protect the key from other (malicious) applications 
    who tries to steal the key.
  \end{solution}

\question[3]
% tags: accountability:E:C:A
What is the purpose of separation of duty?
Explain and illustrate your explanation with an example.

\begin{solution}
  The purpose of separation of duty is to make it more difficult for 
  a malicious entity to subvert the system.
  E.g.\ a systems developer at the bank should not be able to add a back-door 
  for himself so that he later can steal money without anyone noticing.
  If he is only allowed to write the code, but not to certify its correct 
  function --- then there is a higher chance that he is caught before the 
  system is launched.
  So to subvert the system a malicious actor must act together with others.
\end{solution}


\question[3]
% tags: software:E:C:A
Give an example where \enquote{data} can be mistaken for \enquote{code}.

\begin{solution}
  Shell scripting is an easy example.
  Here you can store part of the code in variables, the simply substitute them.
  Consider the following \texttt{/bin/echo -e \$\{1\}}.
  The variable \texttt{\$\{1\}} will be substituted and the result will be 
  interpreted as code.

  A more general example: any time a program interacts with a database using 
  SQL.
  The database cannot differentiate which parts of the SQL-query comes from the 
  program code the programmer wrote and the input provided to that program by 
  the user.
\end{solution}


\question[3]
% tags: ac
% tags: E:C:A
Bell-LaPadula (BLP) is a mandatory access-control (MAC) model for 
confidentiality.
Biba is also a MAC model, but for integrity.
(It is sometimes called \enquote{BLP upside-down}.)

Explain how you can use MAC (e.g.\ Biba) to ensure the integrity of a system, 
e.g.\ to prevent malware from infecting a system.
(Microsoft actually introduced such a technique in Windows Vista.)

\begin{solution}
  We assign different objects to different levels of integrity.
  E.g.\ system files have a high level of integrity, the users documents have 
  lower, but still higher than temporary files.

  When we download files from untrusted sources they will get the lowest level 
  of integrity.
  Thus if we download executable files they will execute with the lowest level 
  of integrity.
  As such they can not modify system files, since those are on a higher level.
  Possibly they cannot modify user-generated files either, since output from 
  local programs (which do not communicate with the outside world) might have a 
  higher level of integrity.
\end{solution}


\question[3]
% tags: passwd:infotheory:usability
% tags: E:C:A
  The National Institute of Standards and Technology (NIST) updated their 
  recommendations on password policies in June 2017.
  Among the changes are:
  \begin{itemize}
    \item Forced password changes should only happen when a breach has 
    occurred.
    \item Long passphrases are favoured over complex passwords, i.e.\ a mix of 
    special characters, upper- and lowercase letters is no longer a 
    requirement.
  \end{itemize}
  Explain why this is a better recommendation.
  Illustrate your explanation with examples.

  \begin{solution}
    The increased security can easily be demonstrated.
    E.g.\ a password of four randomly chosen words forming a passphrase will 
    have \(4\log_2 125\,000 \approx 4\cdot 16.9 \approx 67.7\) bits of entropy.
    (We assume a word list of 125\,000 words, e.g.\ the standard Swedish 
    dictionary.)
    A completely randomly generated password consisting of eight upper- and 
    lowercase letters, numbers and special characters will yield 
    \(8\log_2(26\cdot 2 + 10\cdot 2) \approx 8\cdot 6.2 \approx 49.3\) bits of 
    entropy.

    User will not choose entirely randomly.
    There are studies on how well users choose passwords under these two 
    policies.
    The results favour passphrases.
    In the same study, the conclusion is that passphrases also provide better 
    usability.

    Removing the requirement to periodically change the password further 
    improves usability.
    The periodic change of passwords simply forced users to adapt easy-to-guess 
    systems for their password choices.
  \end{solution}

