\question[3]\label{q:auth:E:C}
  Describe the terms
  \begin{parts}
    \part identification and
    \part authentication.
  \end{parts}
  Make sure to illustrate your explanations by examples.
  You must also give an example of a mechanism for each of the terms.

  \begin{solution}
    In identification you claim an identity.
    This can be done using e.g.~a username, fingerprint or DNA sequence.

    In authentication you prove you are who you claim you are.
    This can be done using e.g.~\emph{who} you are (biometric), \emph{where} 
    you are (location) or what you \emph{do} (biometric), something you 
    \emph{have} (e.g.~BankID), or something you \emph{know} (password).
  \end{solution}


  
\question[3]\label{q:accountability:E:C}
  Separation of duties is a core concept for security.
  \begin{parts}
    \part Describe the two types of separation of duties.
    \part What is the main reason for separation of duties?
  \end{parts}

  \begin{solution}
    There are two types of separation of duties:
    dual control and functional separation.
    Dual control means that two or more subjects must act together (at the same 
    time) to authorize a transaction.
    Functional separation means that several functions are needed to authorize 
    a transaction---e.g.~create a transaction and verify it---and one subject 
    is not allowed to do both functions.

    The reason for separation of duties to make it impossible for one malicious 
    subject to compromise a system.
    With separation of duties the malicious subject must persuade one or more 
    other subjects to collude.
  \end{solution}


  
\question[3]\label{q:sidechannels}
% examgen: sidechannels:E
Give an example of a passive side-channel attack.

\begin{solution}
  The adversary is interested in learning classified information.
  They set up a device which records electromagnetic emissions to reconstruct 
  the image on a screen, thus when a target works with the classified data on 
  the computer the adversary sees the same image.
  This is a passive attack since we only need to record.
\end{solution}


\question[3]
  % tags: foundations:E:C:A
  There are three approaches to security: prevention, detection and reaction.
  Discuss why security is not all about prevention, how do the three approaches 
  complement each other.

  \begin{solution}
    The reason for having these three approaches is partly economy and partly 
    that it is impossible to do prevention for certain things.
    Thus, if we cannot prevent an attack, we must be able to detect it.
    When we have detected it we must be able to recover.

    In some cases it's impossible to recover, however.
    For instance, if the attacker gets the personal data of clients.
    We simply cannot take back this data, there will always be a copy somewhere.
    Thus prevention is the main approach for protecting personal data.
    Prevention in this case comes both in terms of protecting the stored data, 
    but also through data minimization, i.e.\ storing only the necessary data, 
    nothing more.

    In other cases, the recovery might be in terms of insurance paying for the 
    costs of the damage, e.g.\ financial loss.

    In some cases, prevention is possible, but detection and recovery is 
    cheaper.
    For instance, the lunch coupons for restaurants can easily be frauded.
    But this will also be easily detectable, thus the cost of prevention might 
    be higher than the cost of the fraud before detection.

    In other cases, e.g.\ electronic communication, then prevention is cheap --- 
    simply use encryption --- whereas detecting a passive eavesdropper is 
    impossible.
  \end{solution}
\question[3]
  % tags: usability:E:C:A
  Discuss why usability is important for security.

  \begin{solution}
    Då systemen vi är beroende av och som ska upprätthålla vår säkerhet 
    handhas av människor blir psykologin genast viktig.
    Vi behöver psykologin inom säkerhetsområdet för att kunna ta hänsyn till 
    hur människor fungerar när vi konstruerar säkerhetssystem.
    Exempelvis, om vi gör ett system för komplext och användaren tycker att 
    komplexiteten är onödig, då kommer denne användare att aktivt försöka att 
    ta sig runt systemet --- kanske genom att skriva upp långa lösenord 
    istället för att lära sig dem utantill.

    Om vi däremot tar hänsyn till användarnas kognitiva begränsningar, då kan 
    vi konstruera system som både är säkra och enkla att använda.
  \end{solution}
\question[3]\label{q:trustcomp:crypto:E:C}
  A user wishes to provide confidentiality to a file.
  \begin{parts}
    \part She can accomplish this through mechanisms provided in the operating 
    system.
    Explain how this works and what are the limits.

    \part She can also accomplish this through purely cryptographic mechanisms.
    Explain how this works and what are the limits.
  \end{parts}

  \begin{solution}
    The first way she's securing her file is by using access control mechanisms 
    in the operating system (OS).

    Assuming we have physical access to the computer, then we can just read the 
    raw data from the disk.
    This can be accomplished by either booting our own OS on her computer, or 
    by removing the disk.

    If we don't have physical access we can always try to bypass the access 
    control mechanisms in other ways, e.g.\ by gaining privileges in the system 
    or seeing if the OS has failed to protect reading from the raw disk (i.e.\ 
    not using the file system).

    The main point here is that the operating system must be working correctly 
    for its mechanisms to be effective.
    The \emph{running} operating system will provide confidentiality by not 
    allowing other users' requests to open the file.

    The most obvious way to have system independent security for this file is 
    to encrypt it, i.e.~using cryptographic mechanisms.
    This way no one can read it unless they have access to the key, and this is 
    true no matter if you change the environment.
    (Of course, if the system is untrusted someone can get to the key that way, 
    but that's outside the scope of this question.)
  \end{solution}


  
\question[3]\label{q:software}
% tags: software:A
Can a files such as images (e.g.\ JPEGs) and other data be dangerous?

\begin{solution}
  Yes, they can contain machine code which can be executed if there is e.g.\ 
  a buffer overrun vulnerability in the software that reads the data.
\end{solution}



\question[3]
% tags: ac
% tags: E:C
What is mandatory access control?

\begin{solution}
  Mandatory access control sets the access policy for created objects based on 
  fixed rules in the system.
\end{solution}


\question[3]\label{q:passwd:infotheory:E:C:A}
  You are asked to estimate some password policies.
  The policies are the following:
  \begin{description}
    \item[basic12]
      At least 12 characters consisting of upper and lower case, and numbers.
    \item[randswedict4]
      Randomly choose four words from the Dictionary of the Swedish Language 
      (SAOL).
      This dictionary contains approximately 125\,000 words.
  \end{description}
  You should answer the following:
  \begin{parts}
    \part Estimate the entropy for the passsword policies.
    (You may rely on the results in certain published research papers discussed 
    in the course for certain estimates.)
    \part Decide how suitable they are for use in a large organization.
    \part Decide how suitable they are for use in a web application.
  \end{parts}
  Note that you will not get any points without a motivation.


  
