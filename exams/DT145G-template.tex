% Author:  Daniel Bosk <daniel.bosk@miun.se>
\documentclass[svv,addpoints]{miunexam}
\usepackage[utf8]{inputenc}
\usepackage[T1]{fontenc}
\usepackage[swedish,english]{babel}
\usepackage[hyphens]{url}
\usepackage{hyperref}
\usepackage{color}
\usepackage{prettyref,varioref}
\usepackage{subfigure}
\usepackage{amsmath,amssymb}
\usepackage{listings}
\usepackage{authblk}
\usepackage[all]{foreign}

\usepackage{csquotes}
\MakeBlockQuote{<}{|}{>}
\EnableQuotes

\usepackage[natbib,style=alphabetic,maxbibnames=99]{biblatex}
\addbibresource{literature.bib}

%\printanswers

\examtype{Final exam}
\courseid{DT145G}
\course{Computer Security}
\date{<EXAM_DATE>}
\author{%
  Daniel Bosk
}
\affil{%
  Department of Information Systems and Technology,\\
  Mid Sweden University, SE-851\,70 Sundsvall\\
  Email: \href{mailto:daniel.bosk@miun.se}{daniel.bosk@miun.se}\\
  Phone: 010-142\,8709
}

\DeclareMathOperator{\hmac}{HMAC}
\DeclareMathOperator{\xor}{\oplus}
\DeclareMathOperator{\concat}{||}

\begin{document}
\maketitle
\thispagestyle{foot}

\section*{Instructions}%
\label{sec:Instructions}
Carefully read the questions before you start answering them.
Note the time limit of the exam and plan your answers accordingly.
Only answer the question, do not write about subjects remotely related to the
question.

Write your answers on separate sheets, not on the exam paper.
Only write on one side of the sheets.
Start each question on a new sheet.
Do not forget to \emph{motivate your answers.}

Make sure you write your answers clearly, if I cannot read an answer the answer
will be awarded no points---even if the answer is correct.
The questions are \emph{not} sorted by difficulty.

\begin{description}
  \item[Time] 5 hours.
  \item[Aids] Dictionary.
  \item[Questions] \numquestions
\end{description}

\subsection*{Preliminary grades}

Each question can be awarded up to three points: one point for E, two points 
for C and three points for A.

To get an E, you must get at least one point on each question --- \ie no 
question must be awarded zero points.
For a C, you must get two points on at least half of the questions.
For an A, you must get three points on at least half of the questions.

If you get zero on one question, you will get Fx and the possibility for oral 
complementary assessment.
If you get two or more zeroes, you must retake the exam.


\clearpage
\section*{Questions}
The questions are given below.
They are not given in any particular order.

\begin{questions}
  \input{<EXAM_QNAME>-<EXAM_ID>.tex}
\end{questions}


\printbibliography
\end{document}

