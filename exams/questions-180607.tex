\question[3]
% tags: passwd:E:C:A
% tags: infotheory:E:C:A
You've just landed a job at an IT department somewhere and now you're having one 
of your first few days.
There is a discussion in the \enquote{fika room}, the topic is the IT 
department's password policy.

\enquote{Well, every respectable website requires at least eight characters, 
with lower and upper case, numbers and special characters}, the head of 
department says, \enquote{so we have it too}.

What would you like to say in this conversation?

\begin{solution}
The last decades' research in user authentication says that such a policy yields 
bad security.
It forces users to select easy to guess passwords and incentivizes password 
reuse while more secure passwords are disqualified according to the policy.

A better policy is to have at least 12 characters as the only requirement.
Also, no requirement of updating the password at regular intervals --- only if a 
breach has occurred.
\end{solution}
\question[3]
% tags: accountability:E:C:A
What is the purpose of separation of duty?
Explain and illustrate your explanation with an example.

\begin{solution}
  The purpose of separation of duty is to make it more difficult for 
  a malicious entity to subvert the system.
  E.g.\ a systems developer at the bank should not be able to add a back-door 
  for himself so that he later can steal money without anyone noticing.
  If he is only allowed to write the code, but not to certify its correct 
  function --- then there is a higher chance that he is caught before the 
  system is launched.
  So to subvert the system a malicious actor must act together with others.
\end{solution}


\question[3]
  % tags: auth:E:C:A
  Describe the terms
  identification and
  authentication as well as
  how these relate to each other.
  Make sure to illustrate your explanations by examples.

  \begin{solution}
    In identification you claim an identity.
    This can be done using e.g.~a username, fingerprint or DNA sequence.

    In authentication you prove the validity of a claim.
    E.g.\ in the case of identification, prove that you are who you claim you 
    are.
    This can be done using e.g.~\emph{who} you are (biometric), what you 
    \emph{do} (biometric), something you \emph{have} (e.g.~BankID), or something 
    you \emph{know} (password).

    Authentication can also be applied to other attributes than identity, e.g.\ 
    that you are older than a given age limit.
    This can be proven without revealing your identity or birthday.
  \end{solution}


  
\question[3]
% tags: ac
% tags: E:C:A
What is mandatory access control?
Discuss its advantages and disadvantages and its suitability in different 
situations.

\begin{solution}
  Mandatory access control sets the access policy for created objects based on 
  fixed rules in the system.
  Two examples are the Bell-LaPadula and Biba models.
  BLP prevents information flow that can violate confidentiality and Biba 
  prevents information flow that can violate integrity.

  Mandatory access control helps enforcing the policy and to avoid mistakes.
  On the other hand, it will prevent communication downwards (or upwards), 
  which might be necessary sometimes (this is usaully solved by special 
  procedures for declassification).

  However, whether it is good or bad depends on the situation.
  Sometimes it's good to have a combination of mandatory and discretionary 
  access control.
\end{solution}


\question[3]
  % tags: software:E:C:A
  Analyse and compare the three malware reproduction techniques
  virus,
  worm,
  trojan horse.

  \begin{solution}
    The virus inserts its own code into other programs code.
    When the other programs are run the virus' payload is run too and the 
    infection can spread further.
    It requires that programs move between systems to spread from one system to 
    another, e.g.\ by USB-drive or network drives.
    Nowadays, when manual file copying of program executables have decreased, 
    viruses are not good for spreading across systems.

    The worm, on the other hand, spreads by its own means, e.g.\ by utilizing 
    networks (shared file systems, remote executions bugs in network services) 
    or emailing itself using available programs on the computer.
    This is a much better proliferation technique.

    The trojan horse is a legitimate-looking program which contains unwanted 
    functionality.
    E.g.\ it is a flash-light app, but in the background it uploads the contact 
    list to the app's developer.
    The advantage of this class is that the malware will not be suspected, the 
    desired program is running etc., no new processes as for worms.
  \end{solution}



\question[3]
  % tags: usability:crypto:E:C:A
  There are numerous alternatives available today for end-to-end secure 
  communication\footnote{%
    Normally this is referred to as end-to-end \emph{encrypted} communication, 
    but we have integrity in addition to confidentiality.
  }, e.g.\ the apps Signal, WhatsApp and Telegram for instant messaging, media 
  messaging and video calls; PGP and S/MIME for email.
  They are all based on a combination of public-key and shared-key cryptography.

  Discuss the usability challenges facing end-to-end secure communication.

  \begin{solution}
    The challenge for these systems is to provide a design which aligns the 
    security with how users work.

    The easy part is to provide a tool for securing the communication, i.e.\ 
    encrypting and providing integrity checks.

    The hard part is to do proper key management, specifically to authenticate 
    the owners of the other keys.
    The currently most used approach is that of Signal, WhatsApp and Telegram.
    They authenticate phone numbers by sending text messages to the phone 
    number.
    Thus they can be sure that the owner of the phone number is also the owner 
    of the private key.
    (Well, the phone operator can of course impersonate the owner of the phone 
    number.)
    Then another user has identified another user by his or her phone number, 
    then the app will ensure the verified key is used.
  \end{solution}

\question[3]
  % tags: foundations:E:C:A
  There are three approaches to security: prevention, detection and reaction.
  Discuss why security is not all about prevention, how do the three approaches 
  complement each other.

  \begin{solution}
    The reason for having these three approaches is partly economy and partly 
    that it is impossible to do prevention for certain things.
    Thus, if we cannot prevent an attack, we must be able to detect it.
    When we have detected it we must be able to recover.

    In some cases it's impossible to recover, however.
    For instance, if the attacker gets the personal data of clients.
    We simply cannot take back this data, there will always be a copy somewhere.
    Thus prevention is the main approach for protecting personal data.
    Prevention in this case comes both in terms of protecting the stored data, 
    but also through data minimization, i.e.\ storing only the necessary data, 
    nothing more.

    In other cases, the recovery might be in terms of insurance paying for the 
    costs of the damage, e.g.\ financial loss.

    In some cases, prevention is possible, but detection and recovery is 
    cheaper.
    For instance, the lunch coupons for restaurants can easily be frauded.
    But this will also be easily detectable, thus the cost of prevention might 
    be higher than the cost of the fraud before detection.

    In other cases, e.g.\ electronic communication, then prevention is cheap --- 
    simply use encryption --- whereas detecting a passive eavesdropper is 
    impossible.
  \end{solution}
\question[3]\label{q:sidechannels}
% examgen: sidechannels:E
Given an example of an active side-channel attack.

\begin{solution}
  Extracting the secret key from a device by measuring energy consumption or 
  electromagnetic emissions while the device performs computations using the 
  secret key.
  It is an active attack since we might need the device to perform operations 
  on certain ciphertexts (or plaintexts).
\end{solution}


\question[3]
  % tags: trustcomp:crypto:software:E:C:A
  Alice wants to provide confidentiality to a file.
  \begin{parts}
    \part She can accomplish this through mechanisms provided in the operating 
    system.
    Explain how this works and what the limits are.

    \part She can also accomplish this through purely cryptographic mechanisms.
    Explain how this works and what the limits are.
  \end{parts}

  \begin{solution}
    The first way she's securing her file is by using access control mechanisms 
    in the operating system (OS).

    Assuming we have physical access to the computer, then we can just read the 
    raw data from the disk.
    This can be accomplished by either booting our own OS on her computer, or 
    by removing the disk.

    If we don't have physical access we can always try to bypass the access 
    control mechanisms in other ways, e.g.\ by gaining privileges in the system 
    or seeing if the OS has failed to protect reading from the raw disk (i.e.\ 
    not using the file system).

    The main point here is that the operating system must be working correctly 
    for its mechanisms to be effective.
    The \emph{running} operating system will provide confidentiality by not 
    allowing other users' requests to open the file.

    The most obvious way to have system independent security for this file is 
    to encrypt it, i.e.~using cryptographic mechanisms.
    This way no one can read it unless they have access to the key, and this is 
    true no matter if you change the environment.
    (Of course, if the system is untrusted someone can get to the key that way, 
    but that's outside the scope of this question.)
  \end{solution}


  
