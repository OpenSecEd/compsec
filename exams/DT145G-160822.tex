% $Id$
% Author:  Daniel Bosk <daniel.bosk@miun.se>
\documentclass[svv,addpoints]{miunexam}
\usepackage[utf8]{inputenc}
\usepackage[T1]{fontenc}
\usepackage[swedish,english]{babel}
\usepackage[hyphens]{url}
\usepackage{hyperref}
\usepackage{color}
\usepackage{prettyref,varioref}
\usepackage{subfigure}
\usepackage{amsmath,amssymb}
\usepackage{listings}
\usepackage{authblk}

\usepackage{csquotes}
\MakeBlockQuote{<}{|}{>}
\EnableQuotes

\usepackage[natbib,style=alphabetic,maxbibnames=99]{biblatex}
\addbibresource{literature.bib}

\printanswers

\examtype{Final exam}
\courseid{DT145G}
\course{Computer Security}
\date{2016-08-22}
\author{%
  Daniel Bosk
}
\affil{%
  Department of Information and Communication Systems,\\
  Mid Sweden University, SE-851\,70 Sundsvall\\
  Email: \href{mailto:daniel.bosk@miun.se}{daniel.bosk@miun.se}\\
  Phone: 010-142\,8709
}

\DeclareMathOperator{\hmac}{HMAC}
\DeclareMathOperator{\xor}{\oplus}
\DeclareMathOperator{\concat}{||}

\begin{document}
\maketitle
\thispagestyle{foot}

\section*{Instructions}
\label{sec:Instructions}
Carefully read the questions before you start answering them.
Note the time limit of the exam and plan your answers accordingly.
Only answer the question, do not write about subjects remotely related to the
question.

Write your answers on separate sheets, not on the exam paper.
Only write on one side of the sheets.
Start each question on a new sheet.
Do not forget to \emph{motivate your answers.}

Make sure you write your answers clearly, if I cannot read an answer the answer
will be awarded no points---even if the answer is correct.
The questions are \emph{not} sorted by difficulty.

\begin{description}
  \item[Time] 5 hours.
  \item[Aids] Dictionary.
  \item[Maximum points] \numpoints
  \item[Questions] \numquestions
\end{description}

\subsection*{Preliminary grades}
The following grading criteria applies:
E \(\geq 50\%\),
D \(\geq 60\%\),
C \(\geq 70\%\), 
B \(\geq 80\%\),
A \(\geq 90\%\).


\clearpage
\section*{Questions}
The questions are given below.
They are not given in any particular order.

\begin{questions}
\question[2]\label{q:accountability}
What is the purpose of logging?

\begin{solution}
  The purpose of logging is to be able to follow how the system has 
  transitioned between states.
  We want to do this to be able to find vulnerabilities that might have been 
  exploited during a breach.
  Also to verify or reject possible breaches.
\end{solution}


 
\question\label{q:infotheory:passwd:E:C}
  Explain how information theory can be used to estimate the strength of 
  passwords chosen under a given password composition policy:
  \begin{parts}
    \part[2] How can you estimate the upper bound, i.e.~the maximum possible 
    entropy?

    \part[2] Why can't you estimate any (useful) lower bound, i.e.~the minimum 
    possible entropy?

    \part[2] How can you estimate the average case, i.e.~what is usually the 
    case when users choose the passwords?
  \end{parts}

  \begin{solution}
    You assume that every part of the password is chosen uniformly randomly.
    This gives the maximum entropy, i.e.~it is an upper bound.
    You have to account for all choices the password composition policy allows.
    Or rather, all choices the policy removes.

    This is hard because a user can choose a very easy to guess password, which 
    has almost no entropy.
    Similarly, if all users choose the same password, then the entropy would be 
    zero.

    The average case can be estimated as in~\cite{Komanduri2011opa}.
    You have to \emph{sample a lot of user-generated passwords}, then you can 
    perform a frequency analysis to find the probabilities and compute the 
    entropy.
    The users are the stochastic variable (random output) and you must get 
    a large enough sample to estimate the probability distribution.
  \end{solution}


  
\question[3]\label{q:trustcomp}
% tags: trustcomp:E:C:A
Describe the requirements for a process to be able to assess the integrity of 
itself and its execution environment.

\begin{solution}
  If the process can trust its environment (i.e.\ the operating system), then 
  it can rely on the environment to assess its own integrity.
  Thus the process relies on the integrity of the operating system.
  The oerating system in turn relies on the integrity of the hardware and must 
  rely on the hardware to assess its own integrity.
  Hence the process needs hardware that will not allow a modified version of 
  the operating system to run.
\end{solution}



\question\label{q:auth:E:C}
  Describe the terms
  \begin{parts}
    \part[2] identification and
    \part[2] authentication.
  \end{parts}
  Make sure to illustrate your explanations by examples.
  You must also give an example of a mechanism for each of the terms.

  \begin{solution}
    In identification you claim an identity.
    This can be done using e.g.~a username, fingerprint or DNA sequence.

    In authentication you prove you are who you claim you are.
    This can be done using e.g.~\emph{who} you are (biometric), \emph{where} 
    you are (location) or what you \emph{do} (biometric), something you 
    \emph{have} (e.g.~BankID), or something you \emph{know} (password).
  \end{solution}


  
\question\label{q:usability:E:C:A}
  Human psychology is important in security.
  It is used in both security usability and social engineering.
  \begin{parts}
    \part[2] Give an overview of why psychology is important in 
    security.

    \begin{solution}
      Då systemen vi är beroende av och som ska upprätthålla vår säkerhet 
      handhas av människor blir psykologin genast viktig.
      Vi behöver psykologin inom säkerhetsområdet för att kunna ta hänsyn till 
      hur människor fungerar när vi konstruerar säkerhetssystem.
      Exempelvis, om vi gör ett system för komplext och användaren tycker att 
      komplexiteten är onödig, då kommer denne användare att aktivt försöka att 
      ta sig runt systemet --- kanske genom att skriva upp långa lösenord 
      istället för att lära sig dem utantill.

      Om vi däremot tar hänsyn till användarnas kognitiva begränsningar, då kan 
      vi konstruera system som både är säkra och enkla att använda.
    \end{solution}

    \part[4] Give an example of an attack which exploits weaknesses in human 
    psychology.
    Also explain why it works.

    \begin{solution}
      En psykologibaserad attack utnyttjar svagheter hos användarna för att ta 
      sig runt ett säkerhetssystem, det är alltså inte säkerhetssystemen som 
      angrips.

      Ett exempel på en sådan attack kan vara att en användare får ett e-brev 
      som till synes är från banken och som innehåller en länk till en 
      inloggningssida, kallat nätfiske.
      Brevet kan be användaren att uppdatera någonting hos banken via internet.
      Ett förfarande beskrivs och sedan läggs till \enquote{eller klicka på 
        länken}.
      Med en förfarande som låter som att det kan ta fem till tio klick kommer 
      användaren sannolikt att välja enklicksalternativet.
      Notera att förfarandet måste vara korrekt för banken medan länken är till 
      en phishingsida.
      Utformandet kan leda till vad litteraturen~\cite[s. 23]{Anderson2008sea} 
      kallar \emph{\foreignlanguage{english}{capture errors}}, att användaren 
      använder ett invant beteende: i detta fall att användaren klickar på 
      direktlänkar.

      Därutöver försöker nätfiskaren att få användaren att tillämpa fel regler 
      i situationen.
      Exempelvis, användaren kanske (omedvetet) lägger större vikt vid att ett 
      hänglås syns i webbläsaren för säker anslutning än att bankens namn är 
      rätt stavat i URL:en.
      Även att bankens namn finns med någonstans i URL:en kan vara en 
      tillräckligt stark regel för att användaren ska undvika att detektera den 
      felaktiga fiske-URL:en.
    \end{solution}
  \end{parts}


  
\question[3]\label{q:software}
% tags: software:E:C:A
Give an example where \enquote{data} can be mistaken for \enquote{code}.

\begin{solution}
  Shell scripting is an easy example.
  Here you can store part of the code in variables, the simply substitute them.
  Consider the following \texttt{/bin/echo -e \$\{1\}}.
  The variable \texttt{\$\{1\}} will be substituted and the result will be 
  interpreted as code.
\end{solution}



\question\label{q:crypto:foundations:E}
  Explain the following terms:
  \begin{parts}
    \part[1] Confidentiality
    \part[1] Integrity
    \part[1] Availability
    \part[1] Accountability
    \part[1] Non-Repudiation
  \end{parts}

  \begin{solution}
    See~\cite{Gollmann2011cs} and~\cite{Anderson2008sea} for definitions.
  \end{solution}


  
\question[3]\label{q:sidechannels}
% tags: sidechannels:E
Given an example of an active side-channel attack.

\begin{solution}
  Extracting the secret key from a device by measuring energy consumption or 
  electromagnetic emissions while the device performs computations using the 
  secret key.
  It is an active attack since we might need the device to perform operations 
  on certain ciphertexts (or plaintexts).
\end{solution}




\end{questions}


\printbibliography
\end{document}

