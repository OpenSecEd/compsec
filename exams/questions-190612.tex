\question[3]\label{q:software}
% tags: software:trustcomp:E:C:A
Game consoles is a good example of a class of devices where part of the 
security is to lock the user out from the highest privileges.
The classic example is the Sony PlayStation where users at first could choose 
to run Linux instead, but Sony later changed their mind and disallowed that.
However, users found ways around that, namely buffer overruns (\eg stack 
overflows).

Explain, on a conceptual level, how one could exploit a buffer overrun (\eg 
stack overflow) in the PlayStation operating system to install and run Linux 
instead.

\begin{solution}
  The case from reality happened like this:

  Someone found a stack overflow vulnerability in a Zelda game (Twilight 
  princess?).
  One could change the name of the pony in a game save (can be done on another 
  system!) to something very long.
  The system would not check the length of this name since it was \enquote{not 
  changeable}.
  This would cause a stack overflow.
  In the name one could include executable code which would launch the Linux 
  installer.
\end{solution}


\question[3]
  % tags: auth:crypto:E:C:A
  % proto
  Alice and Bob wants to communicate securely, \ie through an end-to-end 
  encrypted and authenticated channel (\eg Signal, Telegram, WhatsApp, 
  PGP/Protonmail).
  What do they have to do to make this happen?

  \begin{solution}
    They would have to create secure keys and store them securely in some 
    devices.
    They would have to verify each other's public key: either they verify it 
    themselves or they must trust someone.
    \Eg they can trust that Signal's verification of phone numbers work and if 
    they previously verified the phone numbers, then it should work now too.
  \end{solution}


  
\question[3]
  % tags: passwd:infotheory:usability:E:C:A
  You are asked to analyse two password policies.
  The policies are the following:
  \begin{description}
    \item[basic12]
      Let \emph{the user choose} at least 12 characters consisting of: upper 
      and lower case, numbers.
    \item[randswedict4]
      \emph{Generate a password for the user} by randomly choosing four words 
      from the Dictionary of the Swedish Language (SAOL).
      This dictionary contains approximately 125\,000 words.
  \end{description}
  Analyse the two policies: what are the advantages and disadvantages of each, 
  how do they compare to each other.
\question[3]
% tags: ac
% tags: E:C:A
What is attribute-based access control (ABAC) and what are its advantages?

\begin{solution}
  It's an access control model.

  It uses attributes in the security policy: e.g.\ identities, age limits, 
  times.
  This requires authenticated attributes.

  This is the most general access control model.
\end{solution}


\question[3]\label{q:accountability}
% tags: accountability:E:C
Explain the idea of double-entry book-keeping.

\begin{solution}
  It originates from banks.
  Every entry is either a credit or a debit.
  Every credit must have a corresponding debit, i.e.\ they cancel each other if 
  added together.
  This means that when all entries are added together, the final balance should 
  be zero.
  Thus, we keep the constant state of zero balance, and when the final balance 
  is non-zero we know that something is wrong.
\end{solution}


